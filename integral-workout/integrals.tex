\documentclass[twoside,10pt,draft]{article}

% Headers for work.
\usepackage{fancyhdr}
\usepackage{enumitem}
\usepackage{graphicx}
\usepackage[margin=1in]{geometry}
\usepackage{setspace}
\usepackage{cite}
\usepackage{subcaption}
\usepackage{multicol}
\usepackage{abstract}

% Math packages.
\usepackage{amsmath}
\usepackage{amssymb}
\usepackage{mathtools}

% Citations.
\usepackage{cite}


% Setup fancyhdr.
\setlength{\headheight}{15.2pt}
\pagestyle{fancy}

% Line spacing.
\singlespacing

\begin{document}
\setlist[description]{font=\space\normalfont\space}

\renewcommand\theequation{\thesection.\arabic{equation}}

\title{Various Cases of Two-Electron Integrals}
\author{Connor Briggs}
\maketitle

\hrule
\paragraph*{Abstract} Two-electron integrals are important for quantum chemistry, and in particular for Hartree-Fock methods. Because of this, there are many ways to compute these. Unfortunately, some methods only work when the four orbitals being integrated have different centers. Therefore, different formulae need to be derived for the cases when multiple orbitals share centers.\\
\hrule


\lhead[]{\thepage}
\chead{Integrals}
\rhead[\thepage]{}

\section{Hermite Polynomials}

For most of the integrals, the kernels can be reduced to a form involving Hermite polynomials. Hermite polynomials are defined using the following equation~\cite{wolfram-hermite}.

\begin{equation}
  H_n(x) e^{-x^2} = \frac{d^n}{dx^n} e^{-x^2}
  \label{herm-def}
\end{equation}
This can be rewritten as a sum~\cite{wolfram-hermite},

\begin{equation}
  H_n(x) = \sum_{k = 0}^{\left\lfloor\frac{n}{2}\right\rfloor} \frac{n!}{k!(n - 2k)!}(-1)^k 2^{n - 2k} x^{n - 2k}
  \label{herm-sum}
\end{equation}
as well as a recurrence relation~\cite{wolfram-hermite}.

\begin{equation}
  H_{n + 1}(x) = 2xH_{n}(x) - 2nH_{n - 1}(x)
  \label{herm-recur}
\end{equation}
From these, a multiple-argument formula can be derived~\cite{wolfram-hermite}.

\begin{equation}
  H_n(\alpha t) = \alpha^n \sum_{k = 0}^{\left\lfloor\frac{n}{2}\right\rfloor}\frac{n!}{k!(n - 2k)!}\left(1 - \frac{1}{\alpha^2}\right)^k H_{n - 2k}(t)
  \label{herm-maf}
\end{equation}

\subsection{Hermite Integrals}

For the two-electron integrals, the following integral becomes useful.

\begin{equation}
  \int_{-\infty}^\infty e^{-\left(a - t\right)^2} H_{n}\left(\alpha t\right) dt
  \label{hermint}
\end{equation}
To solve for equation~\ref{hermint}, note the following, from~\cite{wolfram-hermite}.

\begin{equation}
  \int_{-\infty}^\infty e^{-\left(a - t\right)^2} H_{n}\left(t\right) dt = 2^n \sqrt{\pi} a^n
  \label{base-hermint}
\end{equation}
Then, use equation~\ref{herm-maf} to expand.

\begin{equation}
  \int_{-\infty}^\infty e^{-(a - t)^2} H_n(\alpha t) dt = \int_{-\infty}^\infty \alpha^n e^{-(a - t)^2} \sum_{k = 0}^{\left\lfloor\frac{n}{2}\right\rfloor} \frac{n!}{k!(n - 2k)!}\left(1 - \frac{1}{\alpha^2}\right)^k H_{n - 2k}(t) dt
\end{equation}
Manipulating this equation gives

\begin{equation}
  \int_{-\infty}^\infty e^{-(a - t)^2} H_n(\alpha t) dt = \alpha^n \sum_{k = 0}^{\left\lfloor\frac{n}{2}\right\rfloor} \frac{n!}{k!(n - 2k)!}\left(1 - \frac{1}{\alpha^2}\right)^k \int_{-\infty}^\infty e^{-(a - t)^2} H_{n - 2k}(t) dt
\end{equation}
Using equation~\ref{base-hermint} gives the following.

\begin{equation}
  \int_{-\infty}^\infty e^{-(a - t)^2} H_n(\alpha t) dt = \alpha^n \sum_{k = 0}^{\left\lfloor\frac{n}{2}\right\rfloor} \frac{n!}{k!(n - 2k)!}\left(1 - \frac{1}{\alpha^2}\right)^k 2^{n - 2k} \sqrt{\pi} a^{n - 2k}
  \label{herm-integrated}
\end{equation}
Then, rewrite equation~\ref{herm-integrated} to look more like eqation~\ref{herm-sum}. We need to assume that $\left|\alpha \right| < 1$.
\begin{equation}
  \int_{-\infty}^\infty e^{-(a - t)^2} H_n(\alpha t) dt = \sqrt{\pi} \alpha^n \sum_{k = 0}^{\left\lfloor\frac{n}{2}\right\rfloor} \frac{n!}{k!(n - 2k)!} \left(\frac{1}{\alpha^2} - 1\right)^k (-1)^k 2^{n - k} a^{n - 2k}
\end{equation}
\begin{equation}
  = \sqrt{\pi} \alpha^n \sum_{k = 0}^{\left\lfloor\frac{n}{2}\right\rfloor} \frac{n!}{k!(n - 2k)!} \left(\frac{\alpha^2}{1 - \alpha^2}\right)^{-k} (-1)^k 2^{n - 2k} a^{n - 2k}
\end{equation}


\begin{equation}
  = \sqrt{\pi} \alpha^n \left(\sqrt{\frac{1}{\alpha^2} - 1}\right)^n \sum_{k = 0}^{\left\lfloor\frac{n}{2}\right\rfloor} \frac{n!}{k!(n - 2k)!}\left(\sqrt{\frac{\alpha^2}{1 - \alpha^2}}\right)^{n - 2k} (-1)^k 2^{n - 2k} a^{n - 2k}
  \label{herm-rearranged}
\end{equation}
Finally, using equation~\ref{herm-sum} gives the final value of

\begin{equation}
  \int_{-\infty}^\infty e^{-(a - t)^2} H_n(\alpha t) dt = \sqrt{\pi}\left(\sqrt{1 - \alpha^2}\right)^n H_n\left(a\sqrt{\frac{\alpha^2}{1 - \alpha^2}}\right)
  \label{herm-int-solved}
\end{equation}
  

\subsection{Hermite Orbital Recurrence Relation}

Suppose a Cartesian orbital has the following form.
\begin{equation}
  \psi_{\mathbf{a}} = \left(x - A_x\right)^{a_x}\left(y - A_y\right)^{a_y}\left(z - A_z\right)^{a_z}e^{-\alpha\left(\mathbf{r} - \mathbf{A}\right)^2}
\end{equation}
Also suppose a Hermite orbital has the following form.
\begin{equation}
  \overline{\psi_{\mathbf{p}}} = \zeta^{\frac{p}{2}} H_{p_x}\left(\sqrt{\zeta} \left(x - P_x\right)\right) H_{p_y}\left(\sqrt{\zeta} \left(y - P_y\right)\right)H_{p_z}\left(\sqrt{\zeta} \left(x - P_z\right)\right) e^{-\zeta\left(\mathbf{r} - \mathbf{P}\right)^2}
\end{equation}

Using equation~\ref{herm-recur} gives the following recurrence relation for Hermite orbitals. In all equations after this, $i$ can be $x, y,$ or $z$.
\begin{equation}
  \overline{\psi_{\mathbf{p} + \mathbf{1}_i}} = 2\zeta\left(i - P_i\right)\overline{\psi_{\mathbf{p}}} - 2\zeta p_i\overline{\psi_{\mathbf{p} - \mathbf{1}_i}}
  \label{herm-orb-recur}
\end{equation}
If $\mathbf{A} = \mathbf{P}$, then multiplying by $\psi_{\mathbf{a} - \mathbf{1}_i} / (2\zeta)$ gives the following.
\begin{equation}
  \frac{1}{2\zeta} \psi_{\mathbf{a} - \mathbf{1}_i} \overline{\psi_{\mathbf{p} + \mathbf{1}_i}} = \psi_{\mathbf{a}}\overline{\psi_{\mathbf{p}}} - p_i \psi_{\mathbf{a} - \mathbf{1}_i}\overline{\psi_{\mathbf{p} - \mathbf{1}_i}}
  \label{oc-psi-recur}
\end{equation}
If $\left[a\overline{p}\right|_{u,v} = \frac{(2\beta)^u}{(2\zeta)^v}[a\overline{p}|$, then equation~\ref{oc-psi-recur} gives
\begin{equation}
  \left[\mathbf{a}\overline{\mathbf{p}}\right|_{u,v} = p_i\left[\left(\mathbf{a} - \mathbf{1}_i\right)\overline{\left(\mathbf{p} - \mathbf{1}_i\right)}\right|_{u,v+1} + \left[\left(\mathbf{a} - \mathbf{1}_i\right)\overline{\left(\mathbf{p} + \mathbf{1}_i\right)}\right|_{u,v}
  \label{oc-recur}
\end{equation}
However, if $\mathbf{P} = \frac{\alpha \mathbf{A} + \beta\mathbf{B}}{\zeta}$, then the recurrence relation in equation~\ref{herm-orb-recur} needs to be rewritten as
\begin{equation}
  \overline{\psi_{\mathbf{p} + \mathbf{1}_i}} = 2\zeta\left(i - \frac{\alpha}{\zeta}A_i - \frac{\beta}{\zeta}A_i\right)\overline{\psi_{\mathbf{p}}} + 2\zeta\left(\frac{\beta}{\zeta}A_i - \frac{\beta}{\zeta}B_i\right)\overline{\psi_{\mathbf{p}}} - 2\zeta p_i\overline{\psi_{\mathbf{p} - \mathbf{1}_i}}
\end{equation}

\begin{equation}
  = 2\zeta\left(i - A_i\right)\overline{\psi_{\mathbf{p}}} + 2\beta\left(A_i - B_i\right)\overline{\psi_{\mathbf{p}}} - 2\zeta p_i\overline{\psi_{\mathbf{p} - \mathbf{1}_i}}
  \label{tc-herm-recur}
\end{equation}
Then, multiplying equation~\ref{tc-herm-recur} by $\psi_{\mathbf{a} - \mathbf{1}_i} / (2\zeta)$ gives the following.

\begin{equation}
  \frac{1}{2\zeta}\psi_{\mathbf{a} - \mathbf{1}_i}\overline{\psi_{\mathbf{p} + \mathbf{1}_i}} = \psi_{\mathbf{a}}\overline{\psi_{\mathbf{p}}} + \frac{2\beta}{2\zeta}\left(A_i - B_i\right)\psi_{\mathbf{a} - \mathbf{1}_i}\overline{\psi_{\mathbf{p}}} \\
  - 2p_i\psi_{\mathbf{a} - \mathbf{1}_i}\overline{\psi_{\mathbf{p} - \mathbf{1}_i}}
  \label{tc-herm-orb-recur}
\end{equation}
By letting $[a\overline{p}|_{u,v} = \frac{(2\beta)^u}{(2\zeta)^v}[a\overline{p}|$, equation~\ref{tc-herm-orb-recur} can be rearranged to the following.
\begin{equation}
  \left[\mathbf{a}\overline{\mathbf{p}}\right|_{u,v} = \left[\left(\mathbf{a} - \mathbf{1}_i\right)\overline{\left(\mathbf{p} - \mathbf{1}_i\right)}\right|_{u,v+1} - \left(A_i - B_i\right)\left[\left(\mathbf{a} - \mathbf{1}_i\right)\overline{\mathbf{p}}\right|_{u+1,v+1} + 2p_i\left[\left(\mathbf{a} - \mathbf{1}_i\right)\overline{\left(\mathbf{p} - \mathbf{1}_i\right)}\right|_{u,v}
  \label{tc-recur}
\end{equation}



\section{Two-electron Integrals}
\subsection{One Center}
To find the integral $\left(aa\middle|aa\right)$, start by writing the form of one term.

\begin{multline}
  \left[ab\middle|cd\right] = D_A D_B D_C D_D \iiint\left(x_1 - A_x\right)^{a_x}\left(y_1 - A_y\right)^{a_y} \left(z_1 - A_z\right)^{a_z} e^{-\alpha\left(\mathbf{r}_1 - \mathbf{A}\right)^2} \\
  \left(x_1 - A_x\right)^{b_x}\left(y_1 - A_y\right)^{b_y} \left(z_1 - A_z\right)^{b_z}  e^{-\beta\left(\mathbf{r}_1 - \mathbf{A}\right)^2} \\
  \left(x_2 - A_x\right)^{c_x}\left(y_2 - A_y\right)^{c_y} \left(z_2 - A_z\right)^{c_z} e^{-\gamma\left(\mathbf{r}_2 - \mathbf{A}\right)^2} \\
  \left(x_2 - A_x\right)^{d_x}\left(y_2 - A_y\right)^{d_y} \left(z_2 - A_z\right)^{d_z}  e^{-\delta\left(\mathbf{r}_2 - \mathbf{A}\right)^2} \frac{1}{\left|\mathbf{r}_1 - \mathbf{r}_2\right|} d\mathbf{r_1} d\mathbf{r_2}
  \label{oc-integral}
\end{multline}
Combine terms. Let $\zeta = \alpha + \beta$, $\eta = \gamma + \delta$, $\mathbf{e} = \mathbf{a} + \mathbf{b}$ and $\mathbf{f} = \mathbf{c} + \mathbf{d}$. Also, recenter on $\mathbf{A}$.

\begin{equation}
  \left[e0\middle|f0\right] = \int\int\int\int\int\int x_1^{e_x}y_1^{e_y} z_1^{e_z} x_2^{f_x} y_2^{f_y}z_2^{f_z} e^{-\zeta r_1^2}e^{-\eta r_2^2}\frac{1}{\left|\mathbf{r}_1 - \mathbf{r}_2\right|} d\mathbf{r}_1 d\mathbf{r}_2
\end{equation}
Then, use the recurrence relation in equation~\ref{oc-recur} to give a new integral. Note that due to recentering, $\mathbf{P} = \mathbf{0}$.

\begin{multline}
  \left[\overline{\mathbf{p}}\middle|\overline{\mathbf{q}}\right] = D_A D_B D_C D_D\int\int\int\int\int\int \zeta^{\frac{p}{2}} \eta^{\frac{q}{2}} H_{p_x}\left(\sqrt{\zeta} x_1\right) H_{p_y}\left(\sqrt{\zeta} y_1\right) H_{p_z}\left(\sqrt{\zeta} z_1\right) \\
  H_{q_x}\left(\sqrt{\eta}x_2\right)H_{q_y}\left(\sqrt{\eta}y_2\right)H_{q_z}\left(\sqrt{\eta}z_2\right) e^{-\zeta r_1^2}e^{-\eta r_2^2} \frac{1}{\left|\mathbf{r}_1 - \mathbf{r}_2\right|} d\mathbf{r}_1 d\mathbf{r}_2
  \label{oc-hermite-integral}
\end{multline}
Note the following integral.
\begin{equation}
  \frac{1}{\sqrt{\pi}} \int_{-\infty}^\infty e^{-t^2\left(\mathbf{r}_1 - \mathbf{r}_2\right)^2} dt = \frac{1}{\left|\mathbf{r}_1 - \mathbf{r}_2\right|}
  \label{jesus-kernel}
\end{equation}
Using this means equation~\ref{oc-hermite-integral} can become

\begin{multline}
  \left[\overline{\mathbf{p}}\middle|\overline{\mathbf{q}}\right] = \frac{1}{\sqrt{\pi}} \zeta^{\frac{p}{2}} \eta^{\frac{q}{2}} \int_{-\infty}^\infty dt \int_{-\infty}^\infty \int_{-\infty}^\infty \int_{-\infty}^\infty \int_{-\infty}^\infty \int_{-\infty}^\infty \int_{-\infty}^\infty H_{p_x}\left(\sqrt{\zeta} x_1\right) H_{p_y}\left(\sqrt{\zeta} y_1\right) H_{p_z}\left(\sqrt{\zeta} z_1\right) \\
  H_{q_x}\left(\sqrt{\eta}x_2\right)H_{q_y}\left(\sqrt{\eta}y_2\right)H_{q_z}\left(\sqrt{\eta}z_2\right) e^{-\zeta r_1^2}e^{-\eta r_2^2} e^{-t^2\left(\mathbf{r}_1 - \mathbf{r}_2\right)^2} d\mathbf{r}_1 d\mathbf{r}_2
\end{multline}

Now, solve for the $\mathbf{r}_2$ part of the integral. To do this, combine the Gaussians that include this term.
\begin{equation}
  -\eta r_2^2 - t^2 r_2^2 - t^2 r_1^2 + 2t^2 \mathbf{r}_1 \cdot \mathbf{r}_2
\end{equation}
Complete the square.
\begin{equation}
  -\left(\eta + t^2\right)\left(\mathbf{r}_2 - \frac{t^2}{\eta + t^2}\mathbf{r_1}\right)^2 + \frac{t^4}{\eta + t^2}r_1^2 - t^2r_1^2
\end{equation}
This gives a new integral.
\begin{multline}
  \left[\overline{\mathbf{p}}\middle|\overline{\mathbf{q}}\right] = \frac{1}{\sqrt{\pi}} \zeta^{\frac{p}{2}} \eta^{\frac{q}{2}} \int_{-\infty}^\infty dt \int_{-\infty}^\infty \int_{-\infty}^\infty \int_{-\infty}^\infty H_{p_x}\left(\sqrt{\zeta} x_1\right) H_{p_y}\left(\sqrt{\zeta} y_1\right) H_{p_z}\left(\sqrt{\zeta} z_1\right) e^{-\zeta r_1^2}e^{-\frac{\eta t^2}{\eta + t^2}r_1^2} \\
  \int_{-\infty}^\infty \int_{-\infty}^\infty \int_{-\infty}^\infty H_{q_x}\left(\sqrt{\eta}x_2\right)H_{q_y}\left(\sqrt{\eta}y_2\right)H_{q_z}\left(\sqrt{\eta}z_2\right) e^{-\left(\eta + t^2\right)\left(\mathbf{r}_2 - \frac{t^2}{\eta + t^2}\mathbf{r}_1\right)^2} d\mathbf{r}_2 d\mathbf{r}_1
\end{multline}

Next, use equation~\ref{herm-int-solved} to integrate with respect to $\mathbf{r}_2$. First, let $\mathbf{r}_2' = \sqrt{\eta + t^2}\mathbf{r}_2$, so $d\mathbf{r}_2 = \frac{d\mathbf{r}_2'}{\sqrt{\eta + t^2}}$.
\begin{multline}
  \left[\overline{\mathbf{p}}\middle|\overline{\mathbf{q}}\right] = \frac{1}{\sqrt{\pi}} \zeta^{\frac{p}{2}} \eta^{\frac{q}{2}} \int_{-\infty}^\infty \frac{1}{\sqrt{\eta + t^2}} dt \int_{-\infty}^\infty \int_{-\infty}^\infty \int_{-\infty}^\infty H_{p_x}\left(\sqrt{\zeta} x_1\right) H_{p_y}\left(\sqrt{\zeta} y_1\right) H_{p_z}\left(\sqrt{\zeta} z_1\right) e^{-\zeta r_1^2}e^{-\frac{\eta t^2}{\eta + t^2}r_1^2} \\
  \int_{-\infty}^\infty \int_{-\infty}^\infty \int_{-\infty}^\infty H_{q_x}\left(\sqrt{\frac{\eta}{\eta + t^2}}x_2'\right)H_{q_y}\left(\sqrt{\frac{\eta}{\eta + t^2}}y_2'\right)H_{q_z}\left(\sqrt{\frac{\eta}{\eta + t^2}}z_2'\right) e^{-\left(\mathbf{r}_2' - \frac{t^2}{\sqrt{\eta + t^2}}\mathbf{r}_1\right)^2} d\mathbf{r}_2 d\mathbf{r}_1
\end{multline}

\begin{multline}
  \left[\overline{\mathbf{p}}\middle|\overline{\mathbf{q}}\right] = \pi^{\frac{3}{2}} \frac{1}{\sqrt{\pi}} \zeta^{\frac{p}{2}} \eta^{\frac{q}{2}} \int_{-\infty}^\infty \left(\frac{t^2}{\eta + t^2}\right)^{\frac{q}{2}} \frac{1}{\sqrt{\eta + t^2}} H_{q_x}\left(\sqrt{\frac{\eta}{t^2}}\frac{t^2}{\sqrt{\eta + t^2}} x_1\right) H_{q_y}\left(\sqrt{\frac{\eta}{t^2}}\frac{t^2}{\sqrt{\eta + t^2}} y_1\right)H_{q_z}\left(\sqrt{\frac{\eta}{t^2}}\frac{t^2}{\sqrt{\eta + t^2}} z_1\right)\\
  \int_{-\infty}^\infty \int_{-\infty}^\infty \int_{-\infty}^\infty H_{p_x}\left(\sqrt{\zeta} x_1\right) H_{p_y}\left(\sqrt{\zeta} y_1\right) H_{p_z}\left(\sqrt{\zeta} z_1\right) e^{-\zeta r_1^2}e^{-\frac{\eta t^2}{\eta + t^2}r_1^2} d\mathbf{r}_1 dt  
\end{multline}

Simplifying a bit.
\begin{multline}
  \left[\overline{\mathbf{p}}\middle|\overline{\mathbf{q}}\right] = \pi^{\frac{3}{2}} \frac{1}{\sqrt{\pi}} \zeta^{\frac{p}{2}} \eta^{\frac{q}{2}} \int_{-\infty}^\infty \left(\frac{t^2}{\eta + t^2}\right)^{\frac{q}{2}}\frac{1}{\sqrt{\eta + t^2}}\\
  \int_{-\infty}^\infty \int_{-\infty}^\infty \int_{-\infty}^\infty H_{q_x}\left(\sqrt{\frac{\eta t^2}{\eta + t^2}}x_1\right)H_{q_y}\left(\sqrt{\frac{\eta t^2}{\eta + t^2}} y_1\right)H_{q_z}\left(\sqrt{\frac{\eta t^2}{\eta + t^2}}z_1\right)\\
   H_{p_x}\left(\sqrt{\zeta} x_1\right) H_{p_y}\left(\sqrt{\zeta} y_1\right) H_{p_z}\left(\sqrt{\zeta} z_1\right) e^{-\zeta r_1^2}e^{-\frac{\eta t^2}{\eta + t^2}r_1^2} d\mathbf{r}_1 dt  
\end{multline}
Next, use equation~\ref{herm-def} to rewrite.
\begin{equation}
  H_n(x) e^{-x^2}= \frac{d^n}{dx^n} e^{-x^2}
\end{equation}
Then, let $\sqrt{a}u = x$.
\begin{equation}
  H_n(\sqrt{a}u) e^{-au^2} = \frac{d^n}{dx^n} e^{-au^2} = \left(\frac{d\sqrt{a}u}{du}\right)^n\frac{d^n}{du} e^{-au^2}
\end{equation}

  



\bibliography{integrals}
\bibliographystyle{plain}



\end{document}
