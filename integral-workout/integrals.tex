\documentclass[twoside,10pt,draft]{article}

% Headers for work.
\usepackage{fancyhdr}
\usepackage{enumitem}
\usepackage{graphicx}
\usepackage[margin=1in]{geometry}
\usepackage{setspace}
\usepackage{cite}
\usepackage{subcaption}
\usepackage{multicol}
\usepackage{abstract}

% Math packages.
\usepackage{amsmath}
\usepackage{amssymb}
\usepackage{mathtools}

% Citations.
\usepackage{cite}


% Setup fancyhdr.
\setlength{\headheight}{15.2pt}
\pagestyle{fancy}

% Line spacing.
\singlespacing

\begin{document}
\setlist[description]{font=\space\normalfont\space}

\renewcommand\theequation{\thesection.\arabic{equation}}
\newcommand{\dvar}[1]{\ensuremath{\mathrm{d} #1}}
\newcommand{\dvarb}[1]{\ensuremath{\mathrm{d}\mathbf{#1}}}

\title{Various Cases of Two-Electron Integrals}
\author{Connor Briggs}
\maketitle

\hrule
\paragraph*{Abstract} Two-electron integrals are important for quantum chemistry, and in particular for Hartree-Fock methods. Because of this, there are many ways to compute these. Unfortunately, some methods only work when the four orbitals being integrated have different centers. Therefore, different formulae need to be derived for the cases when multiple orbitals share centers.\\
\hrule


\lhead[]{\thepage}
\chead{Integrals}
\rhead[\thepage]{}

\section{Introduction}

\subsection{Notation}

In the equations below, the following notations are used. $\mathbf{r}$ represents a position vector. Its components are $\mathbf{r} = (x, y, z)$, and its magnitude may be represented normally as $\left|\left|\mathbf{r}\right|\right|$ or $\left|\mathbf{r}\right|$. The square of its magnitude may be represented as $\mathbf{r}^2$ or simply $r^2$. Any subscript on the $r$ will be transferred to its components, so $\mathbf{r}_1 = \left(x_1, y_1, z_1\right)$. Also, the term $\dvarb{r} = \dvar{x}\dvar{y}\dvar{z}$. Letters $a, b, c, d$ will represents Cartesian orbitals, while letters $p, q, r, s$ will represent Hermite orbitals. To avoid confusion, Hermite orbital indices will also be shown with an overbar when in vector form. Capital letters like $\mathbf{A}$ represent orbital centers and have components $\mathbf{A} = \left(A_x, A_y, A_z\right)$. Lower case letters represent angular momentum components. When bold, they represent a vector, such as  $\mathbf{a} = \left(a_x, a_y, a_z\right)$, but when Roman they represent the sum of these components, $a = a_x + a_y + a_z$. When in equations, the letters $a, A, \alpha$ go together; $b, B, \beta$ go together; $c, C \gamma$ go together; and $d, D, \delta$ go together. The vector $\mathbf{1}_i$ represents a unit vector along either the $x$, $y$, or $z$ axis, where $i = x, y, z$.

\section{Hermite Polynomials}

For most of the integrals, the kernels can be reduced to a form involving Hermite polynomials. Hermite polynomials are defined using the following equation~\cite{wolfram-hermite}.

\begin{equation}
  H_n(x) e^{-x^2} = (-1)^n\frac{d^n}{dx^n} e^{-x^2}
  \label{herm-def}
\end{equation}
This definition can be used to show the following.
\begin{equation}
  a^{\frac{n}{2}} H_n\left(x\sqrt{a}\right) e^{-ax^2} = (-1)^n \frac{d^n}{dx^n} e^{-ax^2}
  \label{herm-mult-def}
\end{equation}
This can be rewritten as a sum~\cite{wolfram-hermite},
\begin{equation}
  H_n(x) = \sum_{k = 0}^{\left\lfloor\frac{n}{2}\right\rfloor} \frac{n!}{k!(n - 2k)!}(-1)^k 2^{n - 2k} x^{n - 2k}
  \label{herm-sum}
\end{equation}
as well as a recurrence relation~\cite{wolfram-hermite}.
\begin{equation}
  H_{n + 1}(x) = 2xH_{n}(x) - 2nH_{n - 1}(x)
  \label{herm-recur}
\end{equation}
From these, a multiple-argument formula can be derived~\cite{wolfram-hermite}.
\begin{equation}
  H_n(\alpha t) = \alpha^n \sum_{k = 0}^{\left\lfloor\frac{n}{2}\right\rfloor}\frac{n!}{k!(n - 2k)!}\left(1 - \frac{1}{\alpha^2}\right)^k H_{n - 2k}(t)
  \label{herm-maf}
\end{equation}

\subsection{Hermite Integrals}

For the two-electron integrals, the following integral becomes useful.
\begin{equation}
  \int_{-\infty}^\infty e^{-\left(a - t\right)^2} H_{n}\left(\alpha t\right) dt
  \label{hermint}
\end{equation}
To solve for equation~\ref{hermint}, note the following, from~\cite{wolfram-hermite}.
\begin{equation}
  \int_{-\infty}^\infty e^{-\left(a - t\right)^2} H_{n}\left(t\right) dt = 2^n \sqrt{\pi} a^n
  \label{base-hermint}
\end{equation}
Then, use equation~\ref{herm-maf} to expand.
\begin{equation}
  \int_{-\infty}^\infty e^{-(a - t)^2} H_n(\alpha t) dt = \int_{-\infty}^\infty \alpha^n e^{-(a - t)^2} \sum_{k = 0}^{\left\lfloor\frac{n}{2}\right\rfloor} \frac{n!}{k!(n - 2k)!}\left(1 - \frac{1}{\alpha^2}\right)^k H_{n - 2k}(t) dt
\end{equation}
Manipulating this equation gives
\begin{equation}
  \int_{-\infty}^\infty e^{-(a - t)^2} H_n(\alpha t) dt = \alpha^n \sum_{k = 0}^{\left\lfloor\frac{n}{2}\right\rfloor} \frac{n!}{k!(n - 2k)!}\left(1 - \frac{1}{\alpha^2}\right)^k \int_{-\infty}^\infty e^{-(a - t)^2} H_{n - 2k}(t) dt
\end{equation}
Using equation~\ref{base-hermint} gives the following.
\begin{equation}
  \int_{-\infty}^\infty e^{-(a - t)^2} H_n(\alpha t) dt = \alpha^n \sum_{k = 0}^{\left\lfloor\frac{n}{2}\right\rfloor} \frac{n!}{k!(n - 2k)!}\left(1 - \frac{1}{\alpha^2}\right)^k 2^{n - 2k} \sqrt{\pi} a^{n - 2k}
  \label{herm-integrated}
\end{equation}
Then, rewrite equation~\ref{herm-integrated} to look more like eqation~\ref{herm-sum}. We need to assume that $\left|\alpha \right| < 1$.
\begin{equation}
  \int_{-\infty}^\infty e^{-(a - t)^2} H_n(\alpha t) dt = \sqrt{\pi} \alpha^n \sum_{k = 0}^{\left\lfloor\frac{n}{2}\right\rfloor} \frac{n!}{k!(n - 2k)!} \left(\frac{1}{\alpha^2} - 1\right)^k (-1)^k 2^{n - k} a^{n - 2k}
\end{equation}
\begin{equation}
  = \sqrt{\pi} \alpha^n \sum_{k = 0}^{\left\lfloor\frac{n}{2}\right\rfloor} \frac{n!}{k!(n - 2k)!} \left(\frac{\alpha^2}{1 - \alpha^2}\right)^{-k} (-1)^k 2^{n - 2k} a^{n - 2k}
\end{equation}
\begin{equation}
  = \sqrt{\pi} \alpha^n \left(\sqrt{\frac{1}{\alpha^2} - 1}\right)^n \sum_{k = 0}^{\left\lfloor\frac{n}{2}\right\rfloor} \frac{n!}{k!(n - 2k)!}\left(\sqrt{\frac{\alpha^2}{1 - \alpha^2}}\right)^{n - 2k} (-1)^k 2^{n - 2k} a^{n - 2k}
  \label{herm-rearranged}
\end{equation}
Finally, using equation~\ref{herm-sum} gives the final value of
\begin{equation}
  \int_{-\infty}^\infty e^{-(a - t)^2} H_n(\alpha t) dt = \sqrt{\pi}\left(\sqrt{1 - \alpha^2}\right)^n H_n\left(a\sqrt{\frac{\alpha^2}{1 - \alpha^2}}\right)
  \label{herm-int-solved}
\end{equation}
  
\subsection{Hermite Orbital Recurrence Relation}

Suppose a Cartesian orbital has the following form.
\begin{equation}
  \psi_{\mathbf{a}} = \left(x - A_x\right)^{a_x}\left(y - A_y\right)^{a_y}\left(z - A_z\right)^{a_z}e^{-\alpha\left(\mathbf{r} - \mathbf{A}\right)^2}
\end{equation}
Also suppose a Hermite orbital has the following form.
\begin{equation}
  \overline{\psi_{\mathbf{p}}} = \zeta^{\frac{p}{2}} H_{p_x}\left(\sqrt{\zeta} \left(x - P_x\right)\right) H_{p_y}\left(\sqrt{\zeta} \left(y - P_y\right)\right)H_{p_z}\left(\sqrt{\zeta} \left(x - P_z\right)\right) e^{-\zeta\left(\mathbf{r} - \mathbf{P}\right)^2}
\end{equation}
Using equation~\ref{herm-recur} gives the following recurrence relation for Hermite orbitals. In all equations after this, $i$ can be $x, y,$ or $z$.
\begin{equation}
  \overline{\psi_{\mathbf{p} + \mathbf{1}_i}} = 2\zeta\left(i - P_i\right)\overline{\psi_{\mathbf{p}}} - 2\zeta p_i\overline{\psi_{\mathbf{p} - \mathbf{1}_i}}
  \label{herm-orb-recur}
\end{equation}
If $\mathbf{A} = \mathbf{P}$, then multiplying by $\psi_{\mathbf{a} - \mathbf{1}_i} / (2\zeta)$ gives the following.
\begin{equation}
  \frac{1}{2\zeta} \psi_{\mathbf{a} - \mathbf{1}_i} \overline{\psi_{\mathbf{p} + \mathbf{1}_i}} = \psi_{\mathbf{a}}\overline{\psi_{\mathbf{p}}} - p_i \psi_{\mathbf{a} - \mathbf{1}_i}\overline{\psi_{\mathbf{p} - \mathbf{1}_i}}
  \label{oc-psi-recur}
\end{equation}
If $\left[a\overline{p}\right|_{u,v} = \frac{(2\beta)^u}{(2\zeta)^v}[a\overline{p}|$, then equation~\ref{oc-psi-recur} gives
\begin{equation}
  \left[\mathbf{a}\overline{\mathbf{p}}\right|_{u,v} = p_i\left[\left(\mathbf{a} - \mathbf{1}_i\right)\overline{\left(\mathbf{p} - \mathbf{1}_i\right)}\right|_{u,v} + \left[\left(\mathbf{a} - \mathbf{1}_i\right)\overline{\left(\mathbf{p} + \mathbf{1}_i\right)}\right|_{u,v + 1}
  \label{oc-recur}
\end{equation}
However, if $\mathbf{P} = \frac{\alpha \mathbf{A} + \beta\mathbf{B}}{\zeta}$, then the recurrence relation in equation~\ref{herm-orb-recur} needs to be rewritten as
\begin{equation}
  \overline{\psi_{\mathbf{p} + \mathbf{1}_i}} = 2\zeta\left(i - \frac{\alpha}{\zeta}A_i - \frac{\beta}{\zeta}A_i\right)\overline{\psi_{\mathbf{p}}} + 2\zeta\left(\frac{\beta}{\zeta}A_i - \frac{\beta}{\zeta}B_i\right)\overline{\psi_{\mathbf{p}}} - 2\zeta p_i\overline{\psi_{\mathbf{p} - \mathbf{1}_i}}
\end{equation}
\begin{equation}
  = 2\zeta\left(i - A_i\right)\overline{\psi_{\mathbf{p}}} + 2\beta\left(A_i - B_i\right)\overline{\psi_{\mathbf{p}}} - 2\zeta p_i\overline{\psi_{\mathbf{p} - \mathbf{1}_i}}
  \label{tc-herm-recur}
\end{equation}
Then, multiplying equation~\ref{tc-herm-recur} by $\psi_{\mathbf{a} - \mathbf{1}_i} / (2\zeta)$ gives the following.
\begin{equation}
  \frac{1}{2\zeta}\psi_{\mathbf{a} - \mathbf{1}_i}\overline{\psi_{\mathbf{p} + \mathbf{1}_i}} = \psi_{\mathbf{a}}\overline{\psi_{\mathbf{p}}} + \frac{2\beta}{2\zeta}\left(A_i - B_i\right)\psi_{\mathbf{a} - \mathbf{1}_i}\overline{\psi_{\mathbf{p}}} \\
  - p_i\psi_{\mathbf{a} - \mathbf{1}_i}\overline{\psi_{\mathbf{p} - \mathbf{1}_i}}
  \label{tc-herm-orb-recur}
\end{equation}
By letting $[a\overline{p}|_{u,v} = \frac{(2\beta)^u}{(2\zeta)^v}[a\overline{p}|$, equation~\ref{tc-herm-orb-recur} can be rearranged to the following.
\begin{equation}
  \left[\mathbf{a}\overline{\mathbf{p}}\right|_{u,v} = \left[\left(\mathbf{a} - \mathbf{1}_i\right)\overline{\left(\mathbf{p} + \mathbf{1}_i\right)}\right|_{u,v+1} - \left(A_i - B_i\right)\left[\left(\mathbf{a} - \mathbf{1}_i\right)\overline{\mathbf{p}}\right|_{u+1,v+1} + p_i\left[\left(\mathbf{a} - \mathbf{1}_i\right)\overline{\left(\mathbf{p} - \mathbf{1}_i\right)}\right|_{u,v}
  \label{tc-recur}
\end{equation}


\section{Two-electron Integrals}

For the derivations that follow, it is useful to write the forms of the orbitals being used. A Gaussian orbital centered around the origin has the form

\begin{equation}
  G_{\mathbf{a}}(\mathbf{r}) = \sum_{k = 1}^K D_k x^{a_x} y^{a_y} z^{a_z} e^{-\alpha_k r^2}
  \label{cart-orb}
\end{equation}

A Hermite orbital has a different form.
\begin{equation}
  G_{\overline{\mathbf{p}}}(\mathbf{r}) = \sum_{k = 1}^K D_k \alpha_k^{\frac{p}{2}} H_{p_x}\left(x\sqrt{\alpha_k}\right)H_{p_y}\left(y\sqrt{\alpha_k}\right)H_{p_z}\left(z\sqrt{\alpha_k}\right) e^{-\alpha_k r^2}
  \label{herm-orb}
\end{equation}

\subsection{Four-Centered Integrals}

As a first look at the integrals, consider the full case of four centers. This is the integral that follows.
\begin{multline}
  \left(\mathbf{ab}\middle|\mathbf{cd}\right) = \int_{-\infty}^\infty\int_{-\infty}^\infty\int_{-\infty}^\infty\int_{-\infty}^\infty\int_{-\infty}^\infty\int_{-\infty}^\infty G_{\mathbf{a}}\left(\mathbf{r}_1 - \mathbf{A}\right)G_{\mathbf{b}}\left(\mathbf{r}_1 - \mathbf{B}\right)\\
  G_{\mathbf{c}}\left(\mathbf{r}_2 - \mathbf{C}\right)G_{\mathbf{d}}\left(\mathbf{r}_2 - \mathbf{D}\right)\frac{1}{\left|\mathbf{r}_1 - \mathbf{r}_2\right|} \dvarb{r}_1 \dvarb{r}_2
  \label{fc-full-integral}
\end{multline}
Using the definition of the Cartesian orbital in equation~\ref{cart-orb}, it is possible to derive the following recurrence relation by multiplying the integral by $\frac{x_1 - A_x}{x_1 - B_x} + \frac{A_x - B_x}{x_1 - B_x} = 1$. This can be done for any coordinate.
\begin{equation}
  \left(\mathbf{ab}\middle|\mathbf{cd}\right) = \left(\left(\mathbf{a} + \mathbf{1}_i\right)\left(\mathbf{b} - \mathbf{1}_i\right)\middle|\mathbf{cd}\right) + \left(A_i - B_i\right)\left(\mathbf{a}\left(\mathbf{b} - \mathbf{1}_i\right)\middle|\mathbf{cd}\right)
  \label{fc-recur-1}
\end{equation}
Similarly, by multiplying the integral by $\frac{x_2 - C_x}{x_2 - D_x} + \frac{C_x  - D_x}{x_2 - D_x}$, the other recurrence relation can be derived.
\begin{equation}
  \left(\mathbf{ab}\middle|\mathbf{cd}\right) = \left(\mathbf{ab}\middle|\left(\mathbf{c} + \mathbf{1}_i\right)\left(\mathbf{d} - \mathbf{1}_i\right)\right) + \left(C_i - D_i\right)\left(\mathbf{ab}\middle|\mathbf{c}\left(\mathbf{d} - \mathbf{1}_i\right)\right)
  \label{fc-recur-2}
\end{equation}
These recurrence relations also work termwise. Since this integral is over contracted basis functions, is should be rewritten in terms of uncontracted integrals to more easily manipulate the terms.
\begin{equation}
  \left(\mathbf{ab}\middle|\mathbf{cd}\right) = \sum_{k_A = 1}^{K_A}\sum_{k_B = 1}^{K_B}\sum_{k_C = 1}^{K_C}\sum_{k_D = 1}^{K_D} D_A D_B D_C D_D \left[\mathbf{ab}\middle|\mathbf{cd}\right]
\end{equation}
Note that while the uncontracted integral is not subscripted, it is assumed that its parameters change to the correpsonding values in each of the basis functions that make it up. This uncontracted integral obeys equations~\ref{fc-recur-1} and~\ref{fc-recur-2} as well. The integral of the uncontracted functions has the following form.
\begin{multline}
  \left[\mathbf{ab}\middle|\mathbf{cd}\right] = D_A D_B D_C D_D \int_{-\infty}^\infty\int_{-\infty}^\infty\int_{-\infty}^\infty\int_{-\infty}^\infty\int_{-\infty}^\infty\int_{-\infty}^\infty \left(x_1 - A_x\right)^{a_x}\left(y_1 - A_y\right)^{a_y}\left(z_1 - A_z\right)^{a_z}\\
  \left(x_1 - B_x\right)^{b_x}\left(y_1 - B_y\right)^{b_y}\left(z_1 - B_z\right)^{b_z}\\
  \left(x_2 - C_x\right)^{c_x}\left(y_2 - C_y\right)^{c_y}\left(z_2 - C_z\right)^{c_z}\\
  \left(x_2 - D_x\right)^{d_x}\left(y_2 - D_y\right)^{d_y}\left(z_2 - D_z\right)^{d_z}\\
  e^{-\alpha \left(\mathbf{r}_1 - \mathbf{A}\right)^2} e^{-\beta \left(\mathbf{r}_1 - \mathbf{B}\right)^2} e^{-\gamma \left(\mathbf{r}_2 - \mathbf{C}\right)^2} e^{-\delta \left(\mathbf{r}_2 - \mathbf{D}\right)^2} \frac{1}{\left|\mathbf{r}_1 - \mathbf{r}_2\right|} \dvarb{r}_1 \dvarb{r}_2
\end{multline}
To make this easier to solve, combine the exponents.
\begin{equation}
  \alpha r_1^2 - 2\alpha \mathbf{r}_1 \cdot \mathbf{A} + \alpha A^2 + \beta r_1^2 - 2\beta \mathbf{r}_1 \cdot \mathbf{B} + \beta B^2
\end{equation}
\begin{equation}
  = \left(\alpha + \beta\right)r_1^2 - 2\mathbf{r_1} \cdot \left(\alpha\mathbf{A} + \beta\mathbf{B}\right) + \alpha A^2 + \beta B^2
\end{equation}
Complete the square.
\begin{equation}
  = \left(\alpha + \beta\right)r_1^2 - 2\mathbf{r_1} \cdot \left(\alpha\mathbf{A} + \beta\mathbf{B}\right) + \alpha A^2 + \beta B^2 + \frac{\alpha^2 A^2 + \beta^2 B^2 + 2\alpha\beta\mathbf{A} \cdot \mathbf{B}}{\alpha + \beta} - \frac{\alpha^2 A^2 + \beta^2 B^2 + 2\alpha\beta\mathbf{A} \cdot \mathbf{B}}{\alpha + \beta}
\end{equation}
\begin{equation}
  = \left(\alpha + \beta\right)\left(\mathbf{r}_1 - \frac{\alpha\mathbf{A} + \beta\mathbf{B}}{\alpha + \beta}\right)^2 + \frac{\alpha^2 A^2 + \alpha\beta A^2 + \alpha\beta B^2 + \beta^2B^2 - \alpha^2 A^2 - \beta^2 B^2 - 2\alpha\beta\mathbf{A}\cdot\mathbf{B}}{\alpha + \beta}
\end{equation}
\begin{equation}
  = \left(\alpha + \beta\right)\left(\mathbf{r}_1 - \frac{\alpha\mathbf{A} + \beta\mathbf{B}}{\alpha + \beta}\right)^2 + \frac{\alpha\beta}{\alpha + \beta} \left(\mathbf{A} - \mathbf{B}\right)^2
\end{equation}
Let $\zeta = \alpha + \beta$ and $\mathbf{P} = \frac{\alpha\mathbf{A} + \beta\mathbf{B}}{\zeta}$. Then the equation becomes the following.
\begin{equation}
  = \zeta\left(\mathbf{r}_1 - \mathbf{P}\right)^2 + \frac{\alpha\beta}{\zeta}\left(\mathbf{A} - \mathbf{B}\right)^2
\end{equation}
A similar process can be done for the other two Gaussians. Plugging these in to the integral gives a new integral. For these, let $\eta = \gamma + \delta$ and $\mathbf{Q} = \frac{\gamma\mathbf{C} + \delta\mathbf{D}}{\eta}$.
\begin{multline}
  \left[\mathbf{ab}\middle|\mathbf{cd}\right] = D_A D_B D_C D_D e^{-\frac{\alpha\beta}{\zeta}\left(\mathbf{A} - \mathbf{B}\right)^2} e^{-\frac{\gamma\delta}{\eta}\left(\mathbf{C} - \mathbf{D}\right)^2} \int_{-\infty}^\infty\int_{-\infty}^\infty\int_{-\infty}^\infty\int_{-\infty}^\infty\int_{-\infty}^\infty\int_{-\infty}^\infty\\
  \left(x_1 - A_x\right)^{a_x}\left(y_1 - A_y\right)^{a_y}\left(z_1 - A_z\right)^{a_z} \left(x_1 - B_x\right)^{b_x}\left(y_1 - B_y\right)^{b_y}\left(z_1 - B_z\right)^{b_z}\\
  \left(x_2 - C_x\right)^{c_x}\left(y_2 - C_y\right)^{c_y}\left(z_2 - C_z\right)^{c_z} \left(x_2 - D_x\right)^{d_x}\left(y_2 - D_y\right)^{d_y}\left(z_2 - D_z\right)^{d_z}\\
  e^{-\zeta\left(\mathbf{r}_1 - \mathbf{P}\right)^2} e^{-\eta\left(\mathbf{r}_2 - \mathbf{Q}\right)^2}\frac{1}{\left|\mathbf{r}_1 - \mathbf{r}_2\right|} \dvarb{r}_1 \dvarb{r}_2
\end{multline}
Using the recurrence relations in equations~\ref{fc-recur-1} and~\ref{fc-recur-2}, it is possible to take these integrals down to the form $\left[\mathbf{e0}\middle|\mathbf{f0}\right]$. Then, using the recurrence relation in equation~\ref{tc-recur}, which can be applied to both sides, gives an integral of the following form.
\begin{multline}
  \left[\mathbf{0}\overline{\mathbf{p}}\middle|\mathbf{0}\overline{\mathbf{q}}\right] = \left[\overline{\mathbf{p}}\middle|\overline{\mathbf{q}}\right] = D_A D_B D_C D_D e^{-\frac{\alpha\beta}{\zeta}\left(\mathbf{A} - \mathbf{B}\right)^2}e^{-\frac{\gamma\delta}{\eta}\left(\mathbf{C} - \mathbf{D}\right)^2} \zeta^{\frac{p}{2}}\eta^{\frac{q}{2}}\int_{-\infty}^\infty\int_{-\infty}^\infty\int_{-\infty}^\infty\int_{-\infty}^\infty\int_{-\infty}^\infty\int_{-\infty}^\infty \\
  H_{p_x}\left(\left(x_1 - P_x\right)\sqrt{\zeta}\right)H_{p_y}\left(\left(y_1 - P_y\right)\sqrt{\zeta}\right)H_{p_z}\left(\left(z_1 - P_z\right)\sqrt{\zeta}\right)\\
  H_{q_x}\left(\left(x_2 - Q_x\right)\sqrt{\eta}\right)H_{q_y}\left(\left(y_2 - Q_y\right)\sqrt{\eta}\right)H_{q_z}\left(\left(z_2 - Q_z\right)\sqrt{\eta}\right) \\
  e^{-\zeta\left(\mathbf{r}_1 - \mathbf{P}\right)^2} e^{-\eta\left(\mathbf{r}_2 - \mathbf{Q}\right)^2} \frac{1}{\left|\mathbf{r}_1 - \mathbf{r}_2\right|} \dvarb{r}_1 \dvarb{r}_2
\end{multline}
Note that for the recurrence relation in equation~\ref{tc-recur} to work, the $u,v$ scaled integral needs to be defined as $\left|\mathbf{c}\overline{\mathbf{q}}\right]_{u,v} = \frac{\left(2\delta\right)^u}{\left(2\eta\right)^v}\left|\mathbf{c}\overline{\mathbf{q}}\right]$. Then use the following substitution.
\begin{equation}
  \frac{1}{\left|\mathbf{r}_1 - \mathbf{r}_2\right|} = \frac{1}{\sqrt{\pi}}\int_{-\infty}^\infty e^{-t^2\left(\mathbf{r}_1 - \mathbf{r}_2\right)^2} \dvar{t}
  \label{jesus-kernel}
\end{equation}
The integral then becomes the following.
\begin{multline}
  \left[\overline{\mathbf{p}}\middle|\overline{\mathbf{q}}\right] = D_A D_B D_C D_D e^{-\frac{\alpha\beta}{\zeta}\left(\mathbf{A} - \mathbf{B}\right)^2}e^{-\frac{\gamma\delta}{\eta}\left(\mathbf{C} - \mathbf{D}\right)^2}\frac{\zeta^{\frac{p}{2}}\eta^{\frac{q}{2}}}{\sqrt{\pi}} \int_{-\infty}^\infty\int_{-\infty}^\infty\int_{-\infty}^\infty\int_{-\infty}^\infty\int_{-\infty}^\infty\int_{-\infty}^\infty \int_{-\infty}^\infty\\
  H_{p_x}\left(\left(x_1 - P_x\right)\sqrt{\zeta}\right)H_{p_y}\left(\left(y_1 - P_y\right)\sqrt{\zeta}\right)H_{p_z}\left(\left(z_1 - P_z\right)\sqrt{\zeta}\right)\\
  H_{q_x}\left(\left(x_2 - Q_x\right)\sqrt{\eta}\right)H_{q_y}\left(\left(y_2 - Q_y\right)\sqrt{\eta}\right)H_{q_z}\left(\left(z_2 - Q_z\right)\sqrt{\eta}\right) \\
  e^{-\zeta\left(\mathbf{r}_1 - \mathbf{P}\right)^2} e^{-\eta\left(\mathbf{r}_2 - \mathbf{Q}\right)^2} e^{-t^2 \left(\mathbf{r}_1 - \mathbf{r}_2\right)^2} \dvarb{r}_1 \dvarb{r}_2 \dvar{t}
\end{multline}
Let $\mathbf{r}_1' = \mathbf{r}_1 - \mathbf{P}$ and $\mathbf{r}_2' = \mathbf{r}_2 - \mathbf{Q}$. The integral then becomes the following.
\begin{multline}
  \left[\overline{\mathbf{p}}\middle|\overline{\mathbf{q}}\right] = D_A D_B D_C D_D e^{-\frac{\alpha\beta}{\zeta}\left(\mathbf{A} - \mathbf{B}\right)^2}e^{-\frac{\gamma\delta}{\eta}\left(\mathbf{C} - \mathbf{D}\right)^2}\frac{\zeta^{\frac{p}{2}}\eta^{\frac{q}{2}}}{\sqrt{\pi}} \int_{-\infty}^\infty\int_{-\infty}^\infty\int_{-\infty}^\infty\int_{-\infty}^\infty\int_{-\infty}^\infty\int_{-\infty}^\infty \int_{-\infty}^\infty\\
  H_{p_x}\left(x_1'\sqrt{\zeta}\right)H_{p_y}\left(y_1'\sqrt{\zeta}\right)H_{p_z}\left(z_1'\sqrt{\zeta}\right) H_{q_x}\left(x_2'\sqrt{\eta}\right)H_{q_y}\left(y_2'\sqrt{\eta}\right)H_{q_z}\left(z_2'\sqrt{\eta}\right) \\
  e^{-\zeta\left(\mathbf{r}_1'\right)^2} e^{-\eta\left(\mathbf{r}_2'\right)^2} e^{-t^2 \left(\mathbf{r}_1' - \mathbf{r}_2' + \mathbf{P} - \mathbf{Q}\right)^2} \dvarb{r}_1' \dvarb{r}_2' \dvar{t}
  \label{fc-branch-point}
\end{multline}
Let $\mathbf{R} = \mathbf{Q} - \mathbf{P}$ and combine the Gaussians.
\begin{equation}
  \zeta \left(\mathbf{r}_1'\right)^2 + \eta \left(\mathbf{r}_2'\right)^2 + t^2 \left(\mathbf{r}_1'\right)^2 + t^2 \left(\mathbf{r}_2'\right)^2 + t^2 R^2 - 2t^2 \mathbf{r}_1' \cdot \mathbf{R} + 2t^2 \mathbf{r}_2' \cdot \mathbf{R} - 2t^2 \mathbf{r}_1' \cdot \mathbf{r}_2'
\end{equation}
\begin{equation}
  = \left(\zeta + t^2\right)\left(\mathbf{r}_1'\right)^2 - 2t^2 \mathbf{r}_1' \cdot \left(\mathbf{R} + \mathbf{r}_2'\right) + t^2 R^2 + \left(\eta + t^2\right)\left(\mathbf{r}_2'\right)^2 + 2t^2 \mathbf{r}_2'\cdot\mathbf{R}
\end{equation}
\begin{equation}
  = \left(\zeta + t^2\right)\left(\mathbf{r}_1' - t^2\frac{\mathbf{R} + \mathbf{r}_2'}{\zeta + t^2}\right)^2 + t^2 R^2 + \left(\eta + t^2\right)\left(\mathbf{r}_2'\right)^2 + 2t^2 \mathbf{r}_2'\cdot\mathbf{R} - t^4\frac{R^2 + 2\mathbf{R}\cdot\mathbf{r}_2' + \left(\mathbf{r}_2'\right)^2}{\zeta + t^2}
\end{equation}
\begin{multline}
  = \left(\zeta + t^2\right)\left(\mathbf{r}_1' - t^2\frac{\mathbf{R} + \mathbf{r}_2'}{\zeta + t^2}\right)^2 \\
  + \frac{\zeta t^2 R^2 + t^4 R^2 + \eta\zeta\left(\mathbf{r}_2'\right)^2 + \eta t^2 \left(\mathbf{r}_2'\right)^2 + \zeta t^2 \left(\mathbf{r}_2'\right)^2 + t^4 \left(\mathbf{r}_2'\right)^2 + 2\zeta t^2 \mathbf{r}_2'\cdot \mathbf{R} + 2t^4 \mathbf{r}_2'\cdot\mathbf{R}}{\zeta + t^2} \\
  - \frac{t^4 R^2 + 2t^4 \mathbf{r}_2'\cdot\mathbf{R} + t^4\left(\mathbf{r}_2'\right)^2}{\zeta + t^2}
\end{multline}
\begin{equation}
  = \left(\zeta + t^2\right)\left(\mathbf{r}_1' - t^2\frac{\mathbf{R} + \mathbf{r}_2'}{\zeta + t^2}\right)^2 + \frac{\zeta t^2 R^2 + \eta\zeta \left(\mathbf{r}_2'\right)^2 + \eta t^2 \left(\mathbf{r}_2'\right)^2 + \zeta t^2 \left(\mathbf{r}_2'\right)^2 + 2\zeta t^2 \mathbf{r}_2'\cdot\mathbf{R}}{\zeta + t^2}
\end{equation}
\begin{equation}
  = \left(\zeta + t^2\right)\left(\mathbf{r}_1' - t^2\frac{\mathbf{R} + \mathbf{r}_2'}{\zeta + t^2}\right)^2 + \frac{\left(\eta\zeta + \eta t^2 + \zeta t^2\right)\left(\mathbf{r}_2'\right)^2 + 2\zeta t^2 \mathbf{r}_2'\cdot\mathbf{R} + \zeta t^2 R^2}{\zeta + t^2}
\end{equation}
This means that the integral is now the following.
\begin{multline}
  \left[\overline{\mathbf{p}}\middle|\overline{\mathbf{q}}\right] = D_A D_B D_C D_D e^{-\frac{\alpha\beta}{\zeta}\left(\mathbf{A} - \mathbf{B}\right)^2}e^{-\frac{\gamma\delta}{\eta}\left(\mathbf{C} - \mathbf{D}\right)^2}\frac{\zeta^{\frac{p}{2}}\eta^{\frac{q}{2}}}{\sqrt{\pi}} \int_{-\infty}^\infty\int_{-\infty}^\infty\int_{-\infty}^\infty\int_{-\infty}^\infty\\
  H_{q_x}\left(x_2'\sqrt{\eta}\right)H_{q_y}\left(y_2'\sqrt{\eta}\right)H_{q_z}\left(z_2'\sqrt{\eta}\right) e^{-\frac{\left(\eta\zeta + \eta t^2 + \zeta t^2\right)\left(\mathbf{r}_2'\right)^2 + 2\zeta t^2 \mathbf{r}_2'\cdot\mathbf{R} + \zeta t^2 R^2}{\zeta + t^2}} \\
  \int_{-\infty}^\infty\int_{-\infty}^\infty\int_{-\infty}^\infty H_{p_x}\left(x_1'\sqrt{\zeta}\right)H_{p_y}\left(y_1'\sqrt{\zeta}\right)H_{p_z}\left(z_1'\sqrt{\zeta}\right) e^{- \left(\zeta + t^2\right)\left(\mathbf{r}_1' - t^2\frac{\mathbf{R} + \mathbf{r}_2'}{\zeta + t^2}\right)^2} \dvarb{r}_1' \dvarb{r}_2' \dvar{t}
\end{multline}
Using equation~\ref{herm-int-solved} on the $\mathbf{r}_1'$ component gives the following. Let $\mathbf{u} = \sqrt{\zeta + t^2}\mathbf{r}_1'$, with $\dvarb{r}_1' = \frac{\dvarb{u}}{\sqrt{\zeta + t^2}}$. Note that $1 - \frac{\zeta}{\zeta + t^2} = \frac{t^2}{\zeta + t^2}$ and $\frac{\zeta \left(\zeta + t^2\right)}{t^2 \left(\zeta + t^2\right)} = \frac{\zeta}{t^2}$.
\begin{multline}
  \left[\overline{\mathbf{p}}\middle|\overline{\mathbf{q}}\right] = D_A D_B D_C D_D e^{-\frac{\alpha\beta}{\zeta}\left(\mathbf{A} - \mathbf{B}\right)^2}e^{-\frac{\gamma\delta}{\eta}\left(\mathbf{C} - \mathbf{D}\right)^2}\frac{\zeta^{\frac{p}{2}}\eta^{\frac{q}{2}}}{\sqrt{\pi}} \int_{-\infty}^\infty\int_{-\infty}^\infty\int_{-\infty}^\infty\int_{-\infty}^\infty\\
  H_{q_x}\left(x_2'\sqrt{\eta}\right)H_{q_y}\left(y_2'\sqrt{\eta}\right)H_{q_z}\left(z_2'\sqrt{\eta}\right) e^{-\frac{\left(\eta\zeta + \eta t^2 + \zeta t^2\right)\left(\mathbf{r}_2'\right)^2 + 2\zeta t^2 \mathbf{r}_2'\cdot\mathbf{R} + \zeta t^2 R^2}{\zeta + t^2}} \\
  \frac{\pi^{\frac{3}{2}}}{\left(\zeta + t^2\right)^{\frac{3}{2}}}\left(\frac{t^2}{\zeta + t^2}\right)^{\frac{p}{2}} H_{p_x}\left(t^2 \frac{R_x + x_2'}{\sqrt{\zeta + t^2}}\sqrt{\frac{\zeta}{t^2}}\right)H_{p_y}\left(t^2 \frac{R_y + y_2'}{\sqrt{\zeta + t^2}}\sqrt{\frac{\zeta}{t^2}}\right)H_{p_z}\left(t^2 \frac{R_z + z_2'}{\sqrt{\zeta + t^2}}\sqrt{\frac{\zeta}{t^2}}\right) \dvarb{r}_2' \dvar{t}
\end{multline}
\begin{multline}
  \left[\overline{\mathbf{p}}\middle|\overline{\mathbf{q}}\right] = D_A D_B D_C D_D e^{-\frac{\alpha\beta}{\zeta}\left(\mathbf{A} - \mathbf{B}\right)^2}e^{-\frac{\gamma\delta}{\eta}\left(\mathbf{C} - \mathbf{D}\right)^2}\zeta^{\frac{p}{2}}\eta^{\frac{q}{2}}\pi \int_{-\infty}^\infty\int_{-\infty}^\infty\int_{-\infty}^\infty\int_{-\infty}^\infty\\
  H_{q_x}\left(x_2'\sqrt{\eta}\right)H_{q_y}\left(y_2'\sqrt{\eta}\right)H_{q_z}\left(z_2'\sqrt{\eta}\right) e^{-\frac{\left(\eta\zeta + \eta t^2 + \zeta t^2\right)\left(\mathbf{r}_2'\right)^2 + 2\zeta t^2 \mathbf{r}_2'\cdot\mathbf{R} + \zeta t^2 R^2}{\zeta + t^2}} \\
  \frac{t^p}{\left(\zeta + t^2\right)^{\frac{p + 3}{2}}} H_{p_x}\left(\left(R_x + x_2'\right)\sqrt{\frac{\zeta t^2}{\zeta + t^2}}\right)H_{p_y}\left(\left(R_y + y_2'\right)\sqrt{\frac{\zeta t^2}{\zeta + t^2}}\right)H_{p_z}\left(\left(R_z + z_2'\right)\sqrt{\frac{\zeta t^2}{\zeta + t^2}}\right) \dvarb{r}_2' \dvar{t}
\end{multline}
Now, try to separate the exponent.
\begin{equation}
  \frac{\left(\eta\zeta + \eta t^2 + \zeta t^2\right)\left(\mathbf{r}_2'\right)^2 + 2\zeta t^2 \mathbf{r}_2'\cdot\mathbf{R} + \zeta t^2 R^2}{\zeta + t^2} = \frac{\zeta t^2}{\zeta + t^2} \left(\mathbf{r}_2' + \mathbf{R}\right)^2 + \eta \left(\mathbf{r}_2'\right)^2
\end{equation}
This means that the integral can be rewritten using equation~\ref{herm-mult-def}.
\begin{multline}
  \left[\overline{\mathbf{p}}\middle|\overline{\mathbf{q}}\right] = D_A D_B D_C D_D e^{-\frac{\alpha\beta}{\zeta}\left(\mathbf{A} - \mathbf{B}\right)^2}e^{-\frac{\gamma\delta}{\eta}\left(\mathbf{C} - \mathbf{D}\right)^2}\pi (-1)^{p + q} \int_{-\infty}^\infty\frac{1}{\left(\zeta + t^2\right)^{\frac{3}{2}}} \\
  \int_{-\infty}^\infty\int_{-\infty}^\infty\int_{-\infty}^\infty \left(\frac{\partial^p}{\partial x_2'^{p_x} \partial y_2'^{p_y} \partial z_2'^{p_z}} e^{-\frac{\zeta t^2}{\zeta + t^2}\left(\mathbf{r}_2' + \mathbf{R}\right)^2}\right) \left(\frac{\partial^q}{\partial x_2'^{q_x} \partial y_2'^{q_y} \partial z_2'^{q_z}} e^{-\eta \left(\mathbf{r}_2'\right)^2}\right) \dvarb{r}_2' \dvar{t}
  \label{fc-herm-int-deriv}
\end{multline}
Then, solve the following equation using integration by parts.
\begin{equation}
  (-1)^{n + m} \int_{-\infty}^\infty \left(\frac{\partial^n}{\partial x^n} e^{-a x^2}\right)\left(\frac{\partial^m}{\partial x^m} e^{-b x^2}\right) \dvar{x}
\end{equation}
\begin{equation}
  = \left.(-1)^{n + m}\left(\frac{\partial^n}{\partial x^n} e^{-a x^2}\right)\left(\frac{\partial^{m + 1}}{\partial x^{m + 1}} e^{-b x^2}\right)\right|_{-\infty}^\infty - (-1)^{n + m} \int_{-\infty}^\infty \left(\frac{\partial^{n - 1}}{\partial x^{n - 1}} e^{-a x^2}\right)\left(\frac{\partial^{m + 1}}{\partial x^{m + 1}} e^{-b x^2}\right) \dvar{x}
\end{equation}
\begin{equation}
  = (-1)^{n - 1 + m}\int_{-\infty}^\infty \left(\frac{\partial^{n - 1}}{\partial x^{n - 1}} e^{-a x^2}\right)\left(\frac{\partial^{m + 1}}{\partial x^{m + 1}} e^{-b x^2}\right) \dvar{x}
\end{equation}
This means that ultimately,
\begin{equation}
  (-1)^{m + n}\int_{-\infty}^\infty \left(\frac{\partial^n}{\partial x^n} e^{-a x^2}\right)\left(\frac{\partial^m}{\partial x^m} e^{-b x^2}\right) \dvar{x} = (-1)^m \int_{-\infty}^\infty e^{-ax^2} \left(\frac{\partial^{m + n}}{\partial x^{m + n}} e^{-b x^2}\right) \dvar{x}
\end{equation}
\begin{equation}
  a^{\frac{n}{2}}b^{\frac{m}{2}} \int_{-\infty}^\infty \left(H_n\left(\sqrt{a} x\right) e^{-ax^2}\right) \left(H_m\left(\sqrt{b} x\right) e^{-bx^2}\right) \dvar{x} = b^{\frac{m + n}{2}} (-1)^{n} \int_{-\infty}^\infty H_{n + m}\left(\sqrt{b} x\right) e^{-a x^2} e^{-b x^2} \dvar{x}
  \label{pq-equiv}
\end{equation}
Using this makes equation~\ref{fc-herm-int-deriv} into the following.
\begin{multline}
  \left[\overline{\mathbf{p}}\middle|\overline{\mathbf{q}}\right] = D_A D_B D_C D_D e^{-\frac{\alpha\beta}{\zeta}\left(\mathbf{A} - \mathbf{B}\right)^2}e^{-\frac{\gamma\delta}{\eta}\left(\mathbf{C} - \mathbf{D}\right)^2}\pi (-1)^{q} \int_{-\infty}^\infty \frac{\left(\zeta t^2\right)^{\frac{p + q}{2}}}{\left(\zeta + t^2\right)^{\frac{p + q + 3}{2}}} \\
  \int_{-\infty}^\infty\int_{-\infty}^\infty\int_{-\infty}^\infty e^{-\frac{\zeta t^2}{\zeta + t^2}\left(\mathbf{r}_2' + \mathbf{R}\right)^2} H_{p_x + q_x}\left(\left(x_2' + R_x\right)\sqrt{\frac{\zeta t^2}{\zeta + t^2}}\right)H_{p_y + q_y}\left(\left(y_2' + R_y\right)\sqrt{\frac{\zeta t^2}{\zeta + t^2}}\right)\\
  H_{p_z + q_z}\left(\left(z_2' + R_z\right)\sqrt{\frac{\zeta t^2}{\zeta + t^2}}\right)e^{-\eta\left(\mathbf{r}_2'\right)^2} \dvarb{r}_2' \dvar{t}
\end{multline}
Now, combine the exponents.
\begin{equation}
  \frac{\left(\eta\zeta + \eta t^2 + \zeta t^2\right)\left(\mathbf{r}_2'\right)^2 + 2\zeta t^2 \mathbf{r}_2'\cdot\mathbf{R} + \zeta t^2 R^2}{\zeta + t^2} = \frac{\eta\zeta + \eta t^2 + \zeta t^2}{\zeta + t^2}\left(\mathbf{r}_2'\right)^2 + 2\frac{\zeta t^2}{\zeta + t^2} \mathbf{r}_2' \cdot \mathbf{R} + \frac{\zeta t^2}{\zeta + t^2} R^2
\end{equation}
\begin{equation}
  = \frac{\eta\zeta + \eta t^2 + \zeta t^2}{\zeta + t^2}\left(\mathbf{r}_2' + \frac{\zeta t^2}{\eta\zeta + \eta t^2 + \zeta t^2} \mathbf{R}\right)^2 + \frac{\zeta t^2}{\zeta + t^2} R^2 - \frac{\zeta^2 t^4 R^2}{\left(\zeta + t^2\right)\left(\eta\zeta + \eta t^2 + \zeta t^2\right)}
\end{equation}
\begin{equation}
  = \frac{\eta\zeta + \eta t^2 + \zeta t^2}{\zeta + t^2}\left(\mathbf{r}_2' + \frac{\zeta t^2}{\eta\zeta + \eta t^2 + \zeta t^2} \mathbf{R}\right)^2 + \frac{\eta\zeta^2 t^2 R^2 + \eta\zeta t^4 R^2 + \zeta^2 t^4 R^2 - \zeta^2 t^4 R^2}{\left(\zeta + t^2\right)\left(\eta\zeta + \eta t^2 + \zeta t^2\right)}
\end{equation}
\begin{equation}
  = \frac{\eta\zeta + \eta t^2 + \zeta t^2}{\zeta + t^2}\left(\mathbf{r}_2' + \frac{\zeta t^2}{\eta\zeta + \eta t^2 + \zeta t^2} \mathbf{R}\right)^2 + \frac{\eta\zeta t^2 R^2}{\eta\zeta + \eta t^2 + \zeta t^2}
\end{equation}
This means that the integral becomes the following.
\begin{multline}
  \left[\overline{\mathbf{p}}\middle|\overline{\mathbf{q}}\right] = D_A D_B D_C D_D e^{-\frac{\alpha\beta}{\zeta}\left(\mathbf{A} - \mathbf{B}\right)^2}e^{-\frac{\gamma\delta}{\eta}\left(\mathbf{C} - \mathbf{D}\right)^2}\pi (-1)^{q} \int_{-\infty}^\infty \frac{\left(\zeta t^2\right)^{\frac{p + q}{2}}}{\left(\zeta + t^2\right)^{\frac{p + q + 3}{2}}} e^{-\frac{\eta\zeta t^2 R^2}{\eta\zeta + \eta t^2 + \zeta t^2}}\\
  \int_{-\infty}^\infty\int_{-\infty}^\infty\int_{-\infty}^\infty H_{p_x + q_x}\left(\left(x_2' + R_x\right)\sqrt{\frac{\zeta t^2}{\zeta + t^2}}\right)H_{p_y + q_y}\left(\left(y_2' + R_y\right)\sqrt{\frac{\zeta t^2}{\zeta + t^2}}\right)\\
  H_{p_z + q_z}\left(\left(z_2' + R_z\right)\sqrt{\frac{\zeta t^2}{\zeta + t^2}}\right)e^{-\frac{\eta\zeta + \eta t^2 + \zeta t^2}{\zeta + t^2}\left(\mathbf{r}_2' + \frac{\zeta t^2}{\eta\zeta + \eta t^2 + \zeta t^2} \mathbf{R}\right)^2} \dvarb{r}_2' \dvar{t}
\end{multline}
Let $\mathbf{u} = \sqrt{\frac{\eta\zeta + \eta t^2 + \zeta t^2}{\zeta + t^2}}\mathbf{r}_2'$. The integral becomes the following.
\begin{multline}
  \left[\overline{\mathbf{p}}\middle|\overline{\mathbf{q}}\right] = D_A D_B D_C D_D e^{-\frac{\alpha\beta}{\zeta}\left(\mathbf{A} - \mathbf{B}\right)^2}e^{-\frac{\gamma\delta}{\eta}\left(\mathbf{C} - \mathbf{D}\right)^2}\pi (-1)^{q} \int_{-\infty}^\infty \frac{\left(\zeta t^2\right)^{\frac{p + q}{2}}}{\left(\zeta + t^2\right)^{\frac{p + q}{2}}}\frac{1}{\left(\eta \zeta + \eta t^2 + \zeta t^2\right)^{\frac{3}{2}}} e^{-\frac{\eta\zeta t^2 R^2}{\eta\zeta + \eta t^2 + \zeta t^2}}\\
  \int_{-\infty}^\infty\int_{-\infty}^\infty\int_{-\infty}^\infty H_{p_x + q_x}\left(u_x\sqrt{\frac{\zeta t^2}{\eta\zeta + \zeta t^2 + \eta t^2}} + R_x\sqrt{\frac{\zeta t^2}{\zeta + t^2}}\right)H_{p_y + q_y}\left(u_y\sqrt{\frac{\zeta t^2}{\eta\zeta + \zeta t^2 + \eta t^2}} + R_y\sqrt{\frac{\zeta t^2}{\zeta + t^2}}\right)\\
  H_{p_z + q_z}\left(u_z\sqrt{\frac{\zeta t^2}{\eta\zeta + \zeta t^2 + \eta t^2}} + R_z\sqrt{\frac{\zeta t^2}{\zeta + t^2}}\right) e^{-\left(\mathbf{u} + \frac{\zeta t^2}{\sqrt{\zeta + t^2}}\frac{\mathbf{R}}{\sqrt{\eta\zeta + \eta t^2 + \zeta t^2}}\right)^2} \dvarb{u}\dvar{t}
\end{multline}
Let $\mathbf{v} = \mathbf{u} + \mathbf{R} \sqrt{\frac{\eta\zeta + \zeta t^2 + \eta t^2}{\zeta + t^2}}$. The integral then becomes the following.
\begin{multline}
  \left[\overline{\mathbf{p}}\middle|\overline{\mathbf{q}}\right] = D_A D_B D_C D_D e^{-\frac{\alpha\beta}{\zeta}\left(\mathbf{A} - \mathbf{B}\right)^2}e^{-\frac{\gamma\delta}{\eta}\left(\mathbf{C} - \mathbf{D}\right)^2}\pi (-1)^{q} \int_{-\infty}^\infty \frac{\left(\zeta t^2\right)^{\frac{p + q}{2}}}{\left(\zeta + t^2\right)^{\frac{p + q}{2}}}\frac{1}{\left(\eta \zeta + \eta t^2 + \zeta t^2\right)^{\frac{3}{2}}} e^{-\frac{\eta\zeta t^2 R^2}{\eta\zeta + \eta t^2 + \zeta t^2}}\\
  \int_{-\infty}^\infty\int_{-\infty}^\infty\int_{-\infty}^\infty H_{p_x + q_x}\left(v_x\sqrt{\frac{\zeta t^2}{\eta\zeta + \zeta t^2 + \eta t^2}}\right)H_{p_y + q_y}\left(v_y\sqrt{\frac{\zeta t^2}{\eta\zeta + \zeta t^2 + \eta t^2}}\right)H_{p_z + q_z}\left(v_z\sqrt{\frac{\zeta t^2}{\eta\zeta + \zeta t^2 + \eta t^2}}\right) \\
  e^{-\left(\mathbf{v} - \sqrt{\frac{\eta\zeta + \zeta t^2 + \eta t^2}{\zeta + t^2}} \mathbf{R} + \frac{\zeta t^2}{\sqrt{\zeta + t^2}}\frac{\mathbf{R}}{\sqrt{\eta\zeta + \eta t^2 + \zeta t^2}}\right)^2} \dvarb{v} \dvar{t}
\end{multline}
The exponent can be rewritten.
\begin{equation}
  \sqrt{\frac{\eta\zeta + \zeta t^2 + \eta t^2}{\zeta + t^2}} \mathbf{R} - \frac{\zeta t^2}{\sqrt{\zeta + t^2}}\frac{\mathbf{R}}{\sqrt{\eta\zeta + \eta t^2 + \zeta t^2}} = \frac{\eta\zeta + \zeta t^2 + \eta t^2 - \zeta t^2}{\sqrt{\left(\zeta + t^2\right)\left(\eta\zeta + \eta t^2 + \zeta t^2\right)}}\mathbf{R}
\end{equation}
\begin{equation}
  = \eta \sqrt{\frac{\zeta + t^2}{\eta\zeta + \eta t^2 + \zeta t^2}}\mathbf{R}
\end{equation}
Now, use equation~\ref{herm-int-solved} to transform the integral.
\begin{multline}
  \left[\overline{\mathbf{p}}\middle|\overline{\mathbf{q}}\right] = D_A D_B D_C D_D e^{-\frac{\alpha\beta}{\zeta}\left(\mathbf{A} - \mathbf{B}\right)^2}e^{-\frac{\gamma\delta}{\eta}\left(\mathbf{C} - \mathbf{D}\right)^2}\pi (-1)^{q} \int_{-\infty}^\infty \frac{\left(\zeta t^2\right)^{\frac{p + q}{2}}}{\left(\zeta + t^2\right)^{\frac{p + q}{2}}}\frac{1}{\left(\eta \zeta + \eta t^2 + \zeta t^2\right)^{\frac{3}{2}}} e^{-\frac{\eta\zeta t^2 R^2}{\eta\zeta + \eta t^2 + \zeta t^2}}\\
  \pi^{\frac{3}{2}} \left(\frac{\eta\zeta + \zeta t^2 + \eta t^2 - \zeta t^2}{\eta\zeta + \zeta t^2 + \eta t^2}\right)^{\frac{p + q}{2}}H_{p_x + q_x}\left(\sqrt{\frac{\left(\eta\zeta + \zeta t^2 + \eta t^2\right)\left(\zeta t^2\right)}{\left(\eta\zeta + \eta t^2 + \zeta t^2 - \zeta t^2\right)\left(\eta\zeta + \eta t^2 + \zeta t^2\right)}}\eta\sqrt{\frac{\zeta + t^2}{\eta\zeta + \eta t^2 + \zeta t^2}}R_x\right)\\
  H_{p_y + q_y}\left(\sqrt{\frac{\left(\eta\zeta + \zeta t^2 + \eta t^2\right)\left(\zeta t^2\right)}{\left(\eta\zeta + \eta t^2 + \zeta t^2 - \zeta t^2\right)\left(\eta\zeta + \eta t^2 + \zeta t^2\right)}}\eta\sqrt{\frac{\zeta + t^2}{\eta\zeta + \eta t^2 + \zeta t^2}}R_y\right)\\
  H_{p_z + q_z}\left(\sqrt{\frac{\left(\eta\zeta + \zeta t^2 + \eta t^2\right)\left(\zeta t^2\right)}{\left(\eta\zeta + \eta t^2 + \zeta t^2 - \zeta t^2\right)\left(\eta\zeta + \eta t^2 + \zeta t^2\right)}}\eta\sqrt{\frac{\zeta + t^2}{\eta\zeta + \eta t^2 + \zeta t^2}}R_z\right) \dvar{t}
\end{multline}
\begin{multline}
  \left[\overline{\mathbf{p}}\middle|\overline{\mathbf{q}}\right] = D_A D_B D_C D_D e^{-\frac{\alpha\beta}{\zeta}\left(\mathbf{A} - \mathbf{B}\right)^2}e^{-\frac{\gamma\delta}{\eta}\left(\mathbf{C} - \mathbf{D}\right)^2}\pi^{\frac{5}{2}} (-1)^{q} \int_{-\infty}^\infty \frac{\left(\zeta t^2\right)^{\frac{p + q}{2}}}{\left(\zeta + t^2\right)^{\frac{p + q}{2}}}\frac{1}{\left(\eta \zeta + \eta t^2 + \zeta t^2\right)^{\frac{3}{2}}} e^{-\frac{\eta\zeta t^2 R^2}{\eta\zeta + \eta t^2 + \zeta t^2}}\\
    \left(\frac{\eta\zeta + \eta t^2}{\eta\zeta + \zeta t^2 + \eta t^2}\right)^{\frac{p + q}{2}}H_{p_x + q_x}\left(\eta R_x\sqrt{\frac{\zeta t^2}{\eta\zeta + \eta t^2}}\sqrt{\frac{\zeta + t^2}{\eta\zeta + \eta t^2 + \zeta t^2}}\right)\\
  H_{p_y + q_y}\left(\eta R_y\sqrt{\frac{\zeta t^2}{\eta\zeta + \eta t^2}}\sqrt{\frac{\zeta + t^2}{\eta\zeta + \eta t^2 + \zeta t^2}}\right)H_{p_z + q_z}\left(\eta R_z\sqrt{\frac{\zeta t^2}{\eta\zeta + \eta t^2}}\sqrt{\frac{\zeta + t^2}{\eta\zeta + \eta t^2 + \zeta t^2}}\right) \dvar{t}
\end{multline}
\begin{multline}
  \left[\overline{\mathbf{p}}\middle|\overline{\mathbf{q}}\right] = D_A D_B D_C D_D e^{-\frac{\alpha\beta}{\zeta}\left(\mathbf{A} - \mathbf{B}\right)^2}e^{-\frac{\gamma\delta}{\eta}\left(\mathbf{C} - \mathbf{D}\right)^2}\pi^{\frac{5}{2}} (-1)^{q} \int_{-\infty}^\infty \frac{\left(\eta\zeta t^2\right)^{\frac{p + q}{2}}}{\left(\eta\zeta + \zeta t^2 + \eta t^2\right)^{\frac{p + q}{2}}}\frac{1}{\left(\eta \zeta + \eta t^2 + \zeta t^2\right)^{\frac{3}{2}}} \\
  e^{-\frac{\eta\zeta t^2 R^2}{\eta\zeta + \eta t^2 + \zeta t^2}} H_{p_x + q_x}\left(R_x\sqrt{\frac{\eta\zeta t^2}{\eta\zeta + \eta t^2 + \zeta t^2}}\right)\\
  H_{p_y + q_y}\left(R_y\sqrt{\frac{\eta\zeta t^2}{\eta\zeta + \eta t^2 + \zeta t^2}}\right)H_{p_z + q_z}\left(R_z\sqrt{\frac{\eta\zeta t^2}{\eta\zeta + \eta t^2 + \zeta t^2}}\right) \dvar{t}
\end{multline}
Note that the integrand is even with respect to $t$, so the integral can be rewritten to take this into account.
\begin{multline}
  \left[\overline{\mathbf{p}}\middle|\overline{\mathbf{q}}\right] = D_A D_B D_C D_D e^{-\frac{\alpha\beta}{\zeta}\left(\mathbf{A} - \mathbf{B}\right)^2}e^{-\frac{\gamma\delta}{\eta}\left(\mathbf{C} - \mathbf{D}\right)^2}\pi^{\frac{5}{2}} (-1)^{q} 2\int_{0}^\infty \frac{\left(\eta\zeta t^2\right)^{\frac{p + q}{2}}}{\left(\eta\zeta + \zeta t^2 + \eta t^2\right)^{\frac{p + q}{2}}}\frac{1}{\left(\eta \zeta + \eta t^2 + \zeta t^2\right)^{\frac{3}{2}}} \\
  e^{-\frac{\eta\zeta t^2 R^2}{\eta\zeta + \eta t^2 + \zeta t^2}} H_{p_x + q_x}\left(R_x\sqrt{\frac{\eta\zeta t^2}{\eta\zeta + \eta t^2 + \zeta t^2}}\right)\\
  H_{p_y + q_y}\left(R_y\sqrt{\frac{\eta\zeta t^2}{\eta\zeta + \eta t^2 + \zeta t^2}}\right)H_{p_z + q_z}\left(R_z\sqrt{\frac{\eta\zeta t^2}{\eta\zeta + \eta t^2 + \zeta t^2}}\right) \dvar{t}
\end{multline}
Let $u^2 = \frac{\eta + \zeta}{\eta\zeta}\frac{\eta\zeta t^2}{\eta\zeta + \eta t^2 + \zeta t^2}$. The integral will now go from $0$ to $\lim_{t\rightarrow \infty} \frac{\eta + \zeta}{\eta\zeta}\frac{\eta\zeta t^2}{\eta\zeta + \eta t^2 + \zeta t^2} = 1$. Now, find the derivative of $u$.
\begin{equation}
  2u\frac{\dvar{u}}{\dvar{t}} = \frac{\eta + \zeta}{\eta\zeta} \frac{2\eta\zeta t \left(\eta\zeta + \eta t^2 + \zeta t^2\right) - \eta\zeta t^2 \left(2\eta t + 2\zeta t\right)}{\left(\eta\zeta + \eta t^2 + \zeta t^2\right)^2}
\end{equation}
\begin{equation}
  = \frac{\eta + \zeta}{\eta\zeta} \frac{2\eta^2\zeta^2 t + 2\eta^2 \zeta t^3 + 2\eta \zeta^2 t^3 - 2\eta^2\zeta t^3 - 2\eta \zeta^2 t^3}{\left(\eta\zeta + \eta t^2 + \zeta t^2\right)^2}
\end{equation}
\begin{equation}
  = 2\frac{\eta + \zeta}{\eta\zeta} \frac{\eta^2\zeta^2 t}{\left(\eta\zeta + \eta t^2 + \zeta t^2\right)^2}
\end{equation}
\begin{equation}
  = 2u \sqrt{\frac{\eta + \zeta}{\eta\zeta}} \frac{\eta^{\frac{3}{2}}\zeta^{\frac{3}{2}}}{\left(\eta\zeta + \eta t^2 + \zeta t^2\right)^{\frac{3}{2}}}
\end{equation}
\begin{equation}
  \dvar{t} = \dvar{u}\frac{\left(\eta\zeta + \eta t^2 + \zeta t^2\right)^{\frac{3}{2}}}{\eta\zeta} \sqrt{\frac{1}{\eta + \zeta}}
\end{equation}
The integral is then transformed into the following.
\begin{multline}
  \left[\overline{\mathbf{p}}\middle|\overline{\mathbf{q}}\right] = D_A D_B D_C D_D e^{-\frac{\alpha\beta}{\zeta}\left(\mathbf{A} - \mathbf{B}\right)^2}e^{-\frac{\gamma\delta}{\eta}\left(\mathbf{C} - \mathbf{D}\right)^2}\pi^{\frac{5}{2}} (-1)^{q} 2\int_0^1 \frac{1}{\eta\zeta\sqrt{\eta + \zeta}}\left(\frac{\eta\zeta}{\eta + \zeta}\right)^{\frac{p + q}{2}} u^{p + q} e^{-\frac{\eta\zeta}{\eta + \zeta}u^2 R^2} \\
  H_{p_x + q_x}\left(R_x\sqrt{\frac{\eta\zeta}{\eta + \zeta}}u\right)H_{p_y + q_y}\left(R_y\sqrt{\frac{\eta\zeta}{\eta + \zeta}}u\right)H_{p_z + q_z}\left(R_z\sqrt{\frac{\eta\zeta}{\eta + \zeta}}u\right) \dvar{u}
\end{multline}
Now, let $\overline{\mathbf{r}} = \overline{\mathbf{p} + \mathbf{q}}$ and $\vartheta^2 = \frac{\eta\zeta}{\eta + \zeta}$, and use the notation $\left[\overline{\mathbf{p}}\middle|\overline{\mathbf{q}}\right] = (-1)^q \left[\overline{\mathbf{p} + \mathbf{q}}\right]$.
\begin{multline}
  \left[\overline{\mathbf{r}}\right] = 2D_A D_B D_C D_D e^{-\frac{\alpha\beta}{\zeta}\left(\mathbf{A} - \mathbf{B}\right)^2}e^{-\frac{\gamma\delta}{\eta}\left(\mathbf{C} - \mathbf{D}\right)^2}\frac{\pi^{\frac{5}{2}}}{\eta\zeta\sqrt{\eta + \zeta}} \int_0^1 \left(\vartheta u\right)^r e^{-\vartheta^2 u^2 R^2} \\
  H_{r_x}\left(R_x\vartheta u\right) H_{r_y}\left(R_y\vartheta u\right) H_{r_z}\left(R_z\vartheta u\right) \dvar{u}
\end{multline}
Let $K_P = \frac{\sqrt{2} \pi^{\frac{5}{4}} e^{-\frac{\alpha\beta}{\zeta}\left(\mathbf{A} - \mathbf{B}\right)^2}}{\zeta}$, $K_Q = \frac{\sqrt{2} \pi^{\frac{5}{4}} e^{-\frac{\gamma\delta}{\eta}\left(\mathbf{C} - \mathbf{D}\right)^2}}{\eta}$, $D_P = D_A D_B$, $D_Q = D_C D_D$, and $\omega = \frac{K_PD_P K_Q D_Q}{\sqrt{\eta + \zeta}}$. Then use equation~\ref{herm-recur} to transform the integral.
\begin{equation}
  \left[\overline{\mathbf{r}}\right] = \omega \int_0^1 \left(\vartheta u\right)^r e^{-\vartheta^2 u^2 R^2} \left(2R_x\vartheta u H_{r_x - 1}\left(R_x\vartheta u\right) - 2\left(r_x - 1\right) H_{r_x - 2}\left(R_x \vartheta u\right)\right) H_{r_y}\left(R_y \vartheta u\right) H_{r_z}\left(R_z \vartheta u\right) \dvar{u}
\end{equation}
\begin{multline}
  \left[\overline{\mathbf{r}}\right] = 2\omega R_x \int_0^1 \left(\vartheta u\right)^{r + 1} e^{-\vartheta^2 u^2 R^2} H_{r_x - 1}\left(R_x \vartheta u\right) H_{r_y}\left(R_y \vartheta u\right) H_{r_z}\left(R_z \vartheta u\right) \dvar{u} \\
  - 2\omega \left(r_x - 1\right) \int_0^1 \left(\vartheta u\right)^{r} e^{-\vartheta^2 u^2 R^2} H_{r_x - 2}\left(R_x \vartheta u\right) H_{r_y}\left(R_y \vartheta u\right) H_{r_z}\left(R_z \vartheta u\right) \dvar{u}
\end{multline}
\begin{multline}
  \left[\overline{\mathbf{r}}\right] = 2\omega R_x \int_0^1 \left(\vartheta u\right)^{r - 1 + 2} e^{-\vartheta^2 u^2 R^2} H_{r_x - 1}\left(R_x \vartheta u\right) H_{r_y}\left(R_y \vartheta u\right) H_{r_z}\left(R_z \vartheta u\right) \dvar{u} \\
  - 2\omega \left(r_x - 1\right) \int_0^1 \left(\vartheta u\right)^{r - 2 + 2} e^{-\vartheta^2 u^2 R^2} H_{r_x - 2}\left(R_x \vartheta u\right) H_{r_y}\left(R_y \vartheta u\right) H_{r_z}\left(R_z \vartheta u\right) \dvar{u}
  \label{fc-recur-int}
\end{multline}
Define a new integral.
\begin{equation}
  \left[\overline{\mathbf{r}}\right]^{(m)} = \omega 2^m \int_0^1 \left(\vartheta u\right)^{r + 2m} e^{-\vartheta^2 u^2 R^2} H_{r_x}\left(R_x \vartheta u\right) H_{r_y}\left(R_y \vartheta u\right) H_{r_z}\left(R_z \vartheta u\right) \dvar{u}
\end{equation}
Then using this integral, equation~\ref{fc-recur-int} gives the following recurrence relation.
\begin{equation}
  \left[\overline{\mathbf{r}}\right]^{(m)} = R_i\left[\overline{\mathbf{r} - \mathbf{1}_i}\right]^{(m + 1)} - \left(r_i - 1\right)\left[\overline{\mathbf{r} - \mathbf{2}_i}\right]^{(m + 1)}
  \label{fc-recur-not-int}
\end{equation}
This means that eventually, all of these integrals can be expressed in the form of the following.
\begin{equation}
  \left[\overline{\mathbf{0}}\right]^{(m)} = \omega \left(2\vartheta^2\right)^{m} \int_0^1 u^{2m} e^{-\vartheta^2 u^2 R^2} \dvar{u}
\end{equation}

\subsection{When Centers Match}

Consider equation~\ref{fc-branch-point} for the case of $\mathbf{P} = \mathbf{Q}$. 


  


% \subsection{One Center}
To find the integral $\left(aa\middle|aa\right)$, start by writing the form of one term.
\begin{multline}
  \left[ab\middle|cd\right] = D_A D_B D_C D_D \iiint\left(x_1 - A_x\right)^{a_x}\left(y_1 - A_y\right)^{a_y} \left(z_1 - A_z\right)^{a_z} e^{-\alpha\left(\mathbf{r}_1 - \mathbf{A}\right)^2} \\
  \left(x_1 - A_x\right)^{b_x}\left(y_1 - A_y\right)^{b_y} \left(z_1 - A_z\right)^{b_z}  e^{-\beta\left(\mathbf{r}_1 - \mathbf{A}\right)^2} \\
  \left(x_2 - A_x\right)^{c_x}\left(y_2 - A_y\right)^{c_y} \left(z_2 - A_z\right)^{c_z} e^{-\gamma\left(\mathbf{r}_2 - \mathbf{A}\right)^2} \\
  \left(x_2 - A_x\right)^{d_x}\left(y_2 - A_y\right)^{d_y} \left(z_2 - A_z\right)^{d_z}  e^{-\delta\left(\mathbf{r}_2 - \mathbf{A}\right)^2} \frac{1}{\left|\mathbf{r}_1 - \mathbf{r}_2\right|} d\mathbf{r_1} d\mathbf{r_2}
  \label{oc-integral}
\end{multline}
Combine terms. Let $\zeta = \alpha + \beta$, $\eta = \gamma + \delta$, $\mathbf{e} = \mathbf{a} + \mathbf{b}$ and $\mathbf{f} = \mathbf{c} + \mathbf{d}$. Also, recenter on $\mathbf{A}$.
\begin{equation}
  \left[e0\middle|f0\right] = \int\int\int\int\int\int x_1^{e_x}y_1^{e_y} z_1^{e_z} x_2^{f_x} y_2^{f_y}z_2^{f_z} e^{-\zeta r_1^2}e^{-\eta r_2^2}\frac{1}{\left|\mathbf{r}_1 - \mathbf{r}_2\right|} d\mathbf{r}_1 d\mathbf{r}_2
\end{equation}
Then, use the recurrence relation in equation~\ref{oc-recur} to give a new integral. Note that due to recentering, $\mathbf{P} = \mathbf{0}$.
\begin{multline}
  \left[\overline{\mathbf{p}}\middle|\overline{\mathbf{q}}\right] = D_A D_B D_C D_D\int\int\int\int\int\int \zeta^{\frac{p}{2}} \eta^{\frac{q}{2}} H_{p_x}\left(\sqrt{\zeta} x_1\right) H_{p_y}\left(\sqrt{\zeta} y_1\right) H_{p_z}\left(\sqrt{\zeta} z_1\right) \\
  H_{q_x}\left(\sqrt{\eta}x_2\right)H_{q_y}\left(\sqrt{\eta}y_2\right)H_{q_z}\left(\sqrt{\eta}z_2\right) e^{-\zeta r_1^2}e^{-\eta r_2^2} \frac{1}{\left|\mathbf{r}_1 - \mathbf{r}_2\right|} d\mathbf{r}_1 d\mathbf{r}_2
  \label{oc-hermite-integral}
\end{multline}
Note the following integral.
\begin{equation}
  \frac{1}{\sqrt{\pi}} \int_{-\infty}^\infty e^{-t^2\left(\mathbf{r}_1 - \mathbf{r}_2\right)^2} dt = \frac{1}{\left|\mathbf{r}_1 - \mathbf{r}_2\right|}
  \label{jesus-kernel}
\end{equation}
Using this means equation~\ref{oc-hermite-integral} can become
\begin{multline}
  \left[\overline{\mathbf{p}}\middle|\overline{\mathbf{q}}\right] = \frac{1}{\sqrt{\pi}} \zeta^{\frac{p}{2}} \eta^{\frac{q}{2}} \int_{-\infty}^\infty dt \int_{-\infty}^\infty \int_{-\infty}^\infty \int_{-\infty}^\infty \int_{-\infty}^\infty \int_{-\infty}^\infty \int_{-\infty}^\infty H_{p_x}\left(\sqrt{\zeta} x_1\right) H_{p_y}\left(\sqrt{\zeta} y_1\right) H_{p_z}\left(\sqrt{\zeta} z_1\right) \\
  H_{q_x}\left(\sqrt{\eta}x_2\right)H_{q_y}\left(\sqrt{\eta}y_2\right)H_{q_z}\left(\sqrt{\eta}z_2\right) e^{-\zeta r_1^2}e^{-\eta r_2^2} e^{-t^2\left(\mathbf{r}_1 - \mathbf{r}_2\right)^2} d\mathbf{r}_1 d\mathbf{r}_2
\end{multline}
Now, solve for the $\mathbf{r}_2$ part of the integral. To do this, combine the Gaussians that include this term.
\begin{equation}
  -\eta r_2^2 - t^2 r_2^2 - t^2 r_1^2 + 2t^2 \mathbf{r}_1 \cdot \mathbf{r}_2
\end{equation}
Complete the square.
\begin{equation}
  -\left(\eta + t^2\right)\left(\mathbf{r}_2 - \frac{t^2}{\eta + t^2}\mathbf{r_1}\right)^2 + \frac{t^4}{\eta + t^2}r_1^2 - t^2r_1^2
\end{equation}
This gives a new integral.
\begin{multline}
  \left[\overline{\mathbf{p}}\middle|\overline{\mathbf{q}}\right] = \frac{1}{\sqrt{\pi}} \zeta^{\frac{p}{2}} \eta^{\frac{q}{2}} \int_{-\infty}^\infty dt \int_{-\infty}^\infty \int_{-\infty}^\infty \int_{-\infty}^\infty H_{p_x}\left(\sqrt{\zeta} x_1\right) H_{p_y}\left(\sqrt{\zeta} y_1\right) H_{p_z}\left(\sqrt{\zeta} z_1\right) e^{-\zeta r_1^2}e^{-\frac{\eta t^2}{\eta + t^2}r_1^2} \\
  \int_{-\infty}^\infty \int_{-\infty}^\infty \int_{-\infty}^\infty H_{q_x}\left(\sqrt{\eta}x_2\right)H_{q_y}\left(\sqrt{\eta}y_2\right)H_{q_z}\left(\sqrt{\eta}z_2\right) e^{-\left(\eta + t^2\right)\left(\mathbf{r}_2 - \frac{t^2}{\eta + t^2}\mathbf{r}_1\right)^2} d\mathbf{r}_2 d\mathbf{r}_1
\end{multline}
Next, use equation~\ref{herm-int-solved} to integrate with respect to $\mathbf{r}_2$. First, let $\mathbf{r}_2' = \sqrt{\eta + t^2}\mathbf{r}_2$, so $d\mathbf{r}_2 = \frac{d\mathbf{r}_2'}{\left(\eta + t^2\right)^{\frac{3}{2}}}$.
\begin{multline}
  \left[\overline{\mathbf{p}}\middle|\overline{\mathbf{q}}\right] = \frac{1}{\sqrt{\pi}} \zeta^{\frac{p}{2}} \eta^{\frac{q}{2}} \int_{-\infty}^\infty \frac{1}{\left(\eta + t^2\right)^{\frac{3}{2}}} dt \int_{-\infty}^\infty \int_{-\infty}^\infty \int_{-\infty}^\infty H_{p_x}\left(\sqrt{\zeta} x_1\right) H_{p_y}\left(\sqrt{\zeta} y_1\right) H_{p_z}\left(\sqrt{\zeta} z_1\right) e^{-\zeta r_1^2}e^{-\frac{\eta t^2}{\eta + t^2}r_1^2} \\
  \int_{-\infty}^\infty \int_{-\infty}^\infty \int_{-\infty}^\infty H_{q_x}\left(\sqrt{\frac{\eta}{\eta + t^2}}x_2'\right)H_{q_y}\left(\sqrt{\frac{\eta}{\eta + t^2}}y_2'\right)H_{q_z}\left(\sqrt{\frac{\eta}{\eta + t^2}}z_2'\right) e^{-\left(\mathbf{r}_2' - \frac{t^2}{\sqrt{\eta + t^2}}\mathbf{r}_1\right)^2} d\mathbf{r}_2 d\mathbf{r}_1
\end{multline}
\begin{multline}
  \left[\overline{\mathbf{p}}\middle|\overline{\mathbf{q}}\right] = \pi^{\frac{3}{2}} \frac{1}{\sqrt{\pi}} \zeta^{\frac{p}{2}} \eta^{\frac{q}{2}} \int_{-\infty}^\infty \left(\frac{t^2}{\eta + t^2}\right)^{\frac{q}{2}} \frac{1}{\left(\eta + t^2\right)^{\frac{3}{2}}} H_{q_x}\left(\sqrt{\frac{\eta}{t^2}}\frac{t^2}{\sqrt{\eta + t^2}} x_1\right) H_{q_y}\left(\sqrt{\frac{\eta}{t^2}}\frac{t^2}{\sqrt{\eta + t^2}} y_1\right)\\
  H_{q_z}\left(\sqrt{\frac{\eta}{t^2}}\frac{t^2}{\sqrt{\eta + t^2}} z_1\right)\int_{-\infty}^\infty \int_{-\infty}^\infty \int_{-\infty}^\infty H_{p_x}\left(\sqrt{\zeta} x_1\right) H_{p_y}\left(\sqrt{\zeta} y_1\right) H_{p_z}\left(\sqrt{\zeta} z_1\right) e^{-\zeta r_1^2}e^{-\frac{\eta t^2}{\eta + t^2}r_1^2} d\mathbf{r}_1 dt  
\end{multline}
Simplifying a bit.
\begin{multline}
  \left[\overline{\mathbf{p}}\middle|\overline{\mathbf{q}}\right] = \pi^{\frac{3}{2}} \frac{1}{\sqrt{\pi}} \zeta^{\frac{p}{2}} \eta^{\frac{q}{2}} \int_{-\infty}^\infty \left(\frac{t^2}{\eta + t^2}\right)^{\frac{q}{2}}\frac{1}{\left(\eta + t^2\right)^{\frac{3}{2}}}\\
  \int_{-\infty}^\infty \int_{-\infty}^\infty \int_{-\infty}^\infty H_{q_x}\left(\sqrt{\frac{\eta t^2}{\eta + t^2}}x_1\right)H_{q_y}\left(\sqrt{\frac{\eta t^2}{\eta + t^2}} y_1\right)H_{q_z}\left(\sqrt{\frac{\eta t^2}{\eta + t^2}}z_1\right)\\
   H_{p_x}\left(\sqrt{\zeta} x_1\right) H_{p_y}\left(\sqrt{\zeta} y_1\right) H_{p_z}\left(\sqrt{\zeta} z_1\right) e^{-\zeta r_1^2}e^{-\frac{\eta t^2}{\eta + t^2}r_1^2} d\mathbf{r}_1 dt  
\end{multline}
Next, use equation~\ref{herm-mult-def}.
\begin{multline}
  \left[\overline{\mathbf{p}}\middle|\overline{\mathbf{q}}\right] = (-1)^{p + q} \pi^{\frac{5}{2}} \int_{-\infty}^\infty \frac{1}{\left(\eta + t^2\right)^{\frac{3}{2}}} \int_{-\infty}^\infty \int_{-\infty}^\infty \int_{-\infty}^\infty \left(\frac{\partial^{p}}{\partial x_1^{p_x}\partial y_1^{p_y}\partial z_1^{p_z}}e^{-\zeta r_1^2}\right)\\
  \left(\frac{\partial^{q}}{\partial x_1^{q_x} \partial y_1^{q_y} \partial z_1^{q_z}} e^{-\frac{\eta t^2}{\eta + t^2} r_1^2}\right) d\mathbf{r}_1 dt
  \label{oc-int-diff}
\end{multline}
Using integration by parts and focusing on the $x$ part gives the following.
\begin{multline}
  \int_{-\infty}^\infty \left(\frac{\partial^{p_x}}{\partial x_1^{p_x}} e^{-\zeta x_1^2}\right)\left(\frac{\partial^{q_x}}{\partial x_1^{q_x}} e^{-\frac{\eta t^2}{\eta + t^2} x_1^2}\right) dx_1 = \left.\left(\frac{\partial^{p_x + 1}}{\partial x_1^{p_x + 1}} e^{-\zeta x_1^2}\right)\left(\frac{\partial^{q_x}}{\partial x_1^{q_x}}e^{-\frac{\eta t^2}{\eta + t^2} x_1^2}\right)\right|_{-\infty}^\infty\\
  - \int_{-\infty}^\infty  \left(\frac{\partial^{p_x + 1}}{\partial x_1^{p_x + 1}} e^{-\zeta x_1^2}\right)\left(\frac{\partial^{q_x - 1}}{\partial x_1^{q_x - 1}} e^{-\frac{\eta t^2}{\eta + t^2} x_1^2}\right) dx_1
\end{multline}
The first part goes to zero, so this becomes
\begin{equation}
  \int_{-\infty}^\infty \left(\frac{\partial^{p_x}}{\partial x_1^{p_x}} e^{-\zeta x_1^2}\right)\left(\frac{\partial^{q_x}}{\partial x_1^{q_x}} e^{-\frac{\eta t^2}{\eta + t^2} x_1^2}\right) dx_1 =  - \int_{-\infty}^\infty  \left(\frac{\partial^{p_x + 1}}{\partial x_1^{p_x + 1}} e^{-\zeta x_1^2}\right)\left(\frac{\partial^{q_x - 1}}{\partial x_1^{q_x - 1}} e^{-\frac{\eta t^2}{\eta + t^2} x_1^2}\right) dx_1
\end{equation}
This means that the integral in equation~\ref{oc-int-diff} can be rewritten as the following.
\begin{equation}
  \left[\overline{\mathbf{p}}\middle|\overline{\mathbf{q}}\right] = (-1)^{p} \pi^{\frac{5}{2}} \int_{-\infty}^\infty \frac{1}{\left(\eta + t^2\right)^{\frac{3}{2}}} \int_{-\infty}^\infty \int_{-\infty}^\infty \int_{-\infty}^\infty e^{-\frac{\eta t^2}{\eta + t^2} r_1^2} \left(\frac{\partial^{p + q}}{\partial x_1^{p_x + q_x} \partial y_1^{p_y + q_y} \partial z_1^{p_z + q_z}} e^{-\zeta r_1^2}\right) d\mathbf{r}_1 dt
\end{equation}
\begin{multline}
  \left[\overline{\mathbf{p}}\middle|\overline{\mathbf{q}}\right] = (-1)^{-q} \pi^{\frac{5}{2}} \zeta^{\frac{p + q}{2}} \int_{-\infty}^\infty \frac{1}{\left(\eta + t^2\right)^{\frac{3}{2}}} \int_{-\infty}^\infty \int_{-\infty}^\infty \int_{-\infty}^\infty H_{p_x + q_x}\left(x_1\sqrt{\zeta}\right)H_{p_y + q_y}\left(y_1\sqrt{\zeta}\right)\\
  H_{p_z + q_z}\left(z_1\sqrt{\zeta}\right) e^{-\zeta r_1^2}e^{-\frac{\eta t^2}{\eta + t^2}r_1^2} d\mathbf{r}_1 dt
\end{multline}
Since the kernel to this integral is even with respect to $t$, it can be rewritten as the following.
\begin{multline}
  \left[\overline{\mathbf{p}}\middle|\overline{\mathbf{q}}\right] = (-1)^{-q} \pi^{\frac{5}{2}} \zeta^{\frac{p + q}{2}} 2\int_{-\infty}^\infty \int_{-\infty}^\infty \int_{-\infty}^\infty \int_{0}^\infty \frac{1}{\left(\eta + t^2\right)^{\frac{3}{2}}} H_{p_x + q_x}\left(x_1\sqrt{\zeta}\right)H_{p_y + q_y}\left(y_1\sqrt{\zeta}\right)\\
  H_{p_z + q_z}\left(z_1\sqrt{\zeta}\right) e^{-\zeta r_1^2}e^{-\frac{\eta t^2}{\eta + t^2}r_1^2} dt d\mathbf{r}_1
\end{multline}
Let $\mathbf{u} = \sqrt{\frac{\eta\zeta + \zeta t^2 + \eta t^2}{\eta + t^2}}\mathbf{r}_1$. Then, $d\mathbf{r}_1 = d\mathbf{u}\left(\frac{\eta + t^2}{\eta\zeta + \zeta t^2 + \eta t^2}\right)^{\frac{3}{2}}$.\
\begin{multline}
  \left[\overline{\mathbf{p}}\middle|\overline{\mathbf{q}}\right] = (-1)^{-q} \pi^{\frac{5}{2}} \zeta^{\frac{p + q}{2}} 2\int_{-\infty}^\infty \int_{-\infty}^\infty \int_{-\infty}^\infty \int_{0}^\infty \frac{1}{\left(\eta\zeta + \zeta t^2 + \eta t^2\right)^{\frac{3}{2}}} H_{p_x + q_x}\left(u_x\sqrt{\frac{\eta\zeta + \zeta t^2}{\eta\zeta + \zeta t^2 + \eta t^2}}\right)\\
  H_{p_y + q_y}\left(u_y\sqrt{\frac{\eta\zeta + \zeta t^2}{\eta\zeta + \zeta t^2 + \eta t^2}}\right) H_{p_z + q_z}\left(u_z\sqrt{\frac{\eta\zeta + \zeta t^2}{\eta\zeta + \zeta t^2 + \eta t^2}}\right) e^{-u^2} dt d\mathbf{u}
\end{multline}
Next, using equation~\ref{herm-int-solved} and a formula for the value of a Hermite polynomial at zero gives the following.
\begin{equation}
  \left[\overline{\mathbf{p}}\middle|\overline{\mathbf{q}}\right] = (-1)^{-q} \pi^{\frac{5}{2}} \zeta^{\frac{p + q}{2}} 2\int_{0}^\infty \frac{1}{\left(\eta\zeta + \zeta t^2 + \eta t^2\right)^{\frac{3}{2}}} \pi^{3}\left(\eta t^2\right)^{\frac{p + q}{2}}2^{p + q} \frac{1}{\Gamma\left(\frac{1 - p_x - q_x}{2}\right)\Gamma\left(\frac{1 - p_y - q_y}{2}\right)\Gamma\left(\frac{1 - p_z - q_z}{2}\right)}dt
\end{equation}
Let $u = \sqrt{\frac{\eta t^2 + \zeta t^2}{\eta\zeta + \zeta t^2 + \eta t^2}}$. Then
\begin{equation}
  \left(\eta + \zeta\right)\left(u^2 - 1\right) t^2 + \eta\zeta u^2 = 0
\end{equation}
\begin{equation}
  t = \sqrt{\frac{\eta\zeta u^2}{\left(\eta + \zeta\right)\left(1 - u^2\right)}}
\end{equation}
\begin{equation}
  \frac{du}{dt} = \frac{\sqrt{\eta\zeta + \zeta t^2 + \eta t^2}\left(\left(2\eta t + 2\zeta t\right)\left(\eta\zeta + \zeta t^2 + \eta t^2\right) - \left(\eta t^2 + \zeta t^2\right)\left(2\zeta t + 2\eta t\right)\right)}{2\left(\eta\zeta + \zeta t^2 + \eta t^2\right)^2\sqrt{\eta t^2 + \zeta t^2}}
\end{equation}
\begin{equation}
  \frac{du}{dt} = \frac{\sqrt{\eta\zeta + \zeta t^2 + \eta t^2}\left(\eta^2\zeta t^3 + \eta\zeta^2 t^3\right)}{\left(\eta\zeta + \zeta t^2 + \eta t^2\right)^2\sqrt{\eta t^2 + \zeta t^2}} = \left(\eta\zeta t\right)\frac{\left(\eta t^2 + \zeta t^2\right)^{\frac{1}{2}}}{\left(\eta\zeta + \zeta t^2 + \eta t^2\right)^{\frac{3}{2}}}
\end{equation}
\begin{equation}
  dt = du \frac{\left(\eta\zeta + \zeta t^2 + \eta t^2\right)^{\frac{3}{2}}}{\eta\zeta\sqrt{\eta + \zeta}\frac{\eta\zeta u^2}{\left(\eta + \zeta\right)\left(1 - u^2\right)}}
\end{equation}
The integral then becomes the following.
\begin{equation}
  \left[\overline{\mathbf{p}}\middle|\overline{\mathbf{q}}\right] = (-1)^{-q} \frac{\pi^{\frac{11}{2}} \left(\zeta\eta\right)^{\frac{p + q}{2}} 2^{1 + p + q}}{\Gamma\left(\frac{1 - p_x - q_x}{2}\right)\Gamma\left(\frac{1 - p_y - q_y}{2}\right)\Gamma\left(\frac{1 - p_z - q_z}{2}\right)}\frac{1}{\eta\zeta\sqrt{\eta + \zeta}}\int_0^1 \left(\frac{\eta\zeta u^2}{\left(\eta + \zeta\right)\left(1 - u^2\right)}\right)^{\frac{p + q}{2} - 1} du
\end{equation}


\bibliography{integrals}
\bibliographystyle{plain}


\end{document}
